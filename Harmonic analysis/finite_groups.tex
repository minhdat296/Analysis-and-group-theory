\section{Finite groups}
    \subsection{Sylow subgroups}
        A very basic result in the theory of finite group is one by Lagrange, which tells us that should $G$ be a finite group and $H \leq G$ be a subgroup thereof, then:
            $$\ord(H) \mid \ord(G)$$
        The converse, however, is not true: in general, one does not expect to find a subgroup $H$ of a prescribed order $d \mid \ord(G)$. Fortunately, the next best thing is available, namely that one can still expect $G$ to admit maximal $p$-subgroups. That is to say, if $\ord(G)$ has (positive) prime factorisation:
            $$\ord(G) := \prod_i p_i^{r_i}$$
        then there shall exist subgroups $H_i$ which are of orders $p_i^{r_i}$ and hence maximal amongst the $p_i$-subgroups of $G$. Furthermore, if $H_i, H_i'$ are two maximal $p_i$-subgroups of $G$, then they will even be necessarily conjugate to one another, thus limiting how spreaded out these trees of subgroups of $G$ can be. Finally, it is even known that the number of maximal $p_i$-subgroups of $G$ (so really, the number of conjugacy classes of any given such subgroup) is congruent to $1$ modulo $p_i$.

        In honour of the Norwegian mathematician who discovered the remarkable results listed above, Peter Ludvig Meidell Sylow, maximal $p_i$-subgroups of $G$ are afforded a special name:
        \begin{definition}[Sylow subgroups] \label{def: sylow_subgroups}
            A \textbf{Sylow $p$-subgroup} of a finite group $G$ is a subgroup $H \leq G$ that is maximal amongst all $p$-subgroups of $G$.
        \end{definition}
        \begin{lemma}[A maximality criterion for $p$-subgroups] \label{lemma: maximality_criterion_for_p_subgroups}
            Let $G$ be a finite group and $p$ be a prime. If $H \leq G$ is a $p$-subgroup, then $H$ will be a Sylow $p$-subgroup of $G$ if and only if $\gcd( [G : H], p ) = 1$. 
        \end{lemma}
            \begin{proof}
                Set $\ord(H) := p^n$.
            
                Suppose firstly that $H$ is a Sylow $p$-subgroup. By Lagrange's Theorem, we know that $[G : H] = \frac{\ord(G)}{\ord(H)} = \frac{\ord(G)}{p^n}$, and since $n$ is as large as possible, $[G : H]$ therefore does not contain any power of $p$ as a factor (i.e. there does not exist any $m \geq 1$ such that $p^m \mid [G : H]$), and hence $\gcd( [G : H], p ) = 1$.

                Conversely, suppose that $\gcd([G : H], p) = 1$. This means that there does not exist any $m \geq 1$ such that $p^m \mid [G : H]$. By Lagrange's Theorem again, we know that $\ord(G) = [G : H] \ord(H) = [G : H] p^n$. We see thus that $n$ as large as possible so that $p^n \mid \ord(G)$, and hence $H$ is a Sylow $p$-subgroup by definition. 
            \end{proof}
        \begin{example}[Abelian Sylow $p$-subgroups]
            If $E$ is a finite abelian group then by the Decomposition Theorem for finitely generated abelian groups, we know that:
                $$E \cong \bigoplus_i \Z/p_i^{r_i}$$
            and hence the Sylow $p_i$-subgroups of $E$ shall be isomorphic to $\Z/p_i^{r_i}$. This, in turn, implies that for finite abelian groups, Sylow $p$-subgroups not only exist but for each $p \mid \ord(E)$, there is only one such subgroup up to isomorphisms.
        \end{example}
        \begin{example}[Sylow $p$-subgroups of $\GL_n(\F_q)$]
            Let $p$ be a prime and let $q := p^d$ for some $d \geq 1$, and then consider the finite field $\F_q$ of $q$-elements, obtainable as the splitting field of $x^q - x \in \F_p[x]$.

            Next, consider the group of invertible $n \x n$ matrices with entries coming from $\F_q$, i.e. $\GL_n(\F_q)$. In general, the order of $\GL_n(k)$ for any field $k$ is nothing but the number of possible bases for $k^{\oplus n}$, since an $n \x n$ matrix is invertible if and only if its columns form a basis for $k^{\oplus n}$. Given such a basis:
                $$\{v_1, ..., v_n\} \subset k^{\oplus n}$$
            there are $|k|^n - 1$ ways of choosing $v_1$ since we can choose anything but $0$, $|k|^n - |k|$ to choose $v_2$, etc. and in general, $|k|^n - |k|^{i - 1}$ to choose $v_i$. In total, there are:
                $$(|k|^n - 1)(|k|^n - |k|) ... (|k|^n - |k|^{n - 1}) = \prod_{i = 1}^n (|k|^n - |k|^{i - 1})$$
            ways of choosing bases for $k^{\oplus n}$. We therefore have that:
                $$\ord( \GL_n(\F_q) ) = \prod_{i = 1}^n (q^n - q^{i - 1})$$

            From this, we see that:
                $$\log_p \ord( \GL_n(\F_q) ) = \sum_{i = 1}^n \log_p (q^n - q^{i - 1}) = \sum_{i = 1}^n \left( d(i - 1) + \log_p( q^{n - i - 1} - 1 ) \right) = d \frac{n(n - 1)}{2} + \sum_{i = 1}^n \log_p( q^{n - i - 1} - 1 )$$
            Exponentiating both sides back up will then yield:
                $$\ord( \GL_n(\F_q) ) = p^{ d \frac{n(n - 1)}{2} } \prod_{i = 1}^n (q^{n - i - 1} - 1)$$
            All of the factors $q^{n - i - 1} - 1$ are coprime to $p$, so any Sylow $p$-subgroup of $\GL_n(\F_q)$ must be of order $p^{ d\frac{n(n - 1)}{2} }$, and $\GL_n(\F_q)$ can not have $p'$-subgroups for any other prime $p' \not = p$.

            One can then also prove that $\GL_n(\F_q)$ actually admits at least one Sylow $p$-subgroup by showing that the group $B^+(\F_q)$ of invertible upper triangular $n \x n$ matrices whose diagonal entries are $1$.
        \end{example}
        The fact that $\GL_n(\F_q)$ admits a Sylow $p$-subgroup will later on be important for showing that Sylow $p$-subgroups exist also for general finite groups. Our argument will rely on the fact that any finite group can be embedded as a subgroup into some $\GL_n(\F_q)$, for some $n \gg 1$. First, let us show that any symmetric group can be embedded into some $\GL_n(\F_q)$, and then exploit the fact that every finite group $G$ embeds as a subgroup into the subgroup $S_{\ord(G)}$ if we let $G$ act on itself via left/right-multiplication.
        \begin{proposition}[Symmetric groups as matrix groups] \label{prop: symmetric_groups_as_matrix_groups}
            Fix a positive integer $n$ along with some finite field $\F_q$. Then, the symmetric group $S_n$ admits a faithful representation:
                $$\rho: S_n \to \GL_n(\F_p)$$
            which realises $S_n$ as a subgroup of $\GL_n(\F_q)$.
        \end{proposition}
            \begin{proof}
                Choose a basis $\{v_1, ..., v_n\} \subset \F_q^{\oplus n}$ and then let the elements $\sigma \in S_n$ act on this basis by:
                    $$\rho(\sigma) \cdot v_i := v_{\sigma(i)}$$
                i.e. by permuting the basis vectors. This clearly extends to an $\F_q$-linear action on $\F_q^{\oplus n}$, so the only left to show is that $\rho$ is injective. For this, let us note firstly that because basis vectors are non-zero by definition, we have that:
                    $$v_{\sigma(i)} \not = 0$$
                for all $1 \leq i \leq n$. Next, because if $v = \sum_{i = 1}^n a_i v_i$ then $\rho(\sigma) \cdot v = \sum_{i = 1}^n a_i v_{\sigma(i)}$, which vanishes if and only if $a_i = 0$ for all $1 \leq i \leq n$, since each $v_{\sigma(i)}$ is a basis vector, and hence if and only if $v = 0$. This proves injectivity.
            \end{proof}
        \begin{corollary}[Finite groups as matrix groups] \label{coro: finite_groups_as_matrix_groups}
            Any finite group $G$ has a faithful representation:
                $$\rho_G: G \to \GL_{\ord(G)}(\F_q)$$
            for any finite field $\F_q$.
        \end{corollary}
            \begin{proof}
                Embed $G$ as a subgroup into $S_{\ord(G)}$ and then apply proposition \ref{prop: symmetric_groups_as_matrix_groups}.
            \end{proof}
        
        \begin{lemma}[Conjugacies of Sylow subgroups] \label{lemma: conjugacies_of_sylow_subgroups}
            Let $G$ be a finite group and $p$ be a prime. If $H$ is a Sylow $p$-subgroup of $G$, then for any $g \in G$, $\Ad_G(g) \cdot H$ will also be a Sylow $p$-subgroup of $G$.
        \end{lemma}
            \begin{proof}
                
            \end{proof}

    \subsection{Characters theory}