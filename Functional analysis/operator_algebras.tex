\section{Operator algebras}
    \begin{convention}
        Throughout this entire subsection, we work within the tensor category $\bbC\Ban$ of complex Banach spaces. In particular, terms like \say{Banach space} or \say{Banach algebra} shall always be understood to be over $\bbC$. 
    \end{convention}
    \begin{convention} \label{conv: banach_algebras_are_associative}
        Banach algebras are never assumed to be either unital (e.g. convolution of $L^1$-functions is not unital) or commutative unless specifically stated to be so, but they shall always be associative.
    \end{convention}
    
    \subsection{Operator spectra and \texorpdfstring{$\rmC^*$}{}-algebras}
        \begin{example}
            Before we begin our study of $\rmC^*$-algebras, let us point out certain common complex Banach algebras:
            \begin{itemize}
                \item \textbf{(Bounded continuous functions):} If $X$ is a compact topological space then the $\bbC$-algebra:
                    $$C^0(X) := \Maps(X, \bbC)$$
                of complex-valued continuous functions on $X$ will be a Banach algebra with respect to the sup-norm, given by $\norm{f}_{\sup} := \sup_{x \in X} |f(x)|$. When $X$ is non-compact, we can not apply the Extreme Value Theorem to guarantee that $\sup_{x \in X} |f(x)|$ shall always be a finite non-negative real number, and therefore $\norm{-}_{\sup}: X \to \R_{\geq 0}$ might fail to be a norm; of course, one can always restrict one's attention to the $\bbC$-subalgebra:
                    $$C^b(X) := \Maps^b(X, \bbC)$$
                of bounded complex-valued continuous functions on $X$ (i.e. functions $f: X \to \bbC$ such that $|f(x)| < +\infty$ for all $x \in X$) in order to obtain a Banach algebra with respect to the sup-norm: now, even if $X$ is non-compact, $\norm{-}_{\sup}: X \to \R_{\geq 0}$ will still never return infinite values, meaning that we can then verify that it is indeed a norm and that $C^b(X)$ is topologically complete with respect to it to check that it is a Banach algebra. Of course, we have $C^0(X) = C^b(X)$ when $X$ is compact. 
                \item \textbf{(Bounded operators):} More generally, for $X$ an arbitrary topological space and $V$ a Banach space, one might consider the complex vector space:
                    $$C^b(X, V) := \Maps^b(X, V)$$
                of bounded $V$-valued continuous functions on $X$ and check that it is a Banach space; of course, when $X$ is compact, one can consider instead the entire space $C^0(X, \bbC) := \Maps(X, V)$ of $V$-valued continuous functions on $X$.
                
                A particular case of this phenomenon is that of bounded continuous linear maps between topological vector spaces: if $W$ is a topological vector space and $V$ is as before, then the $\bbC$-vector space $\Hom_{\cont}(W, V)$ of continuous $\bbC$-linear maps from $W$ to $V$ (i.e. $\bbC$-linear maps $\varphi: W \to V$ such that $\sup_{w \in W} \frac{\norm{\varphi(w)}}{\norm{w}} < +\infty$) shall be a Banach subspace of $\Maps^b(W, V)$. This - in turn - implies that for any Banach space $V$, the space $\End_{\cont}^b(V)$ of bounded continuous operators on $V$ is a Banach algebra with respect to the operator norm (given by $\norm{\varphi} := \sup_{w \in W} \frac{\norm{\varphi(w)}}{\norm{w}}$). 
                \item \textbf{($L^1$-functions on locally compact groups):} Let $G$ be a locally compact topological group, $\mu$ a fixed Haar measure thereon, and denote by $L^1(G, \mu)$ the corresponding space of absolutely integrable complex-valued functions, always implicitly understood to be equipped with the $L^1$-norm, given by $\norm{f}_{L^1(G, \mu)} := \int_G |f| d\mu$. The operation of convolution then endows $L^1(G, \mu)$ with the structure of a \textit{non-unital} Banach algebra.
            \end{itemize}
        \end{example}
        
        \begin{definition}[Spectra of operators] \label{def: spectra_of_operators}
            Let $A$ be a (complex) unital Banach algebra. Then the \textbf{spectrum} of any element $x \in A$ shall be the following closed (in fact, compact) subset of $\bbC$:
                $$\sigma(x) := \{\lambda \in \bbC \mid \lambda - x \not \in A^{\x}\}$$
        \end{definition}
        \begin{example}[Spectral elements are eigenvalues] \label{example: spectral_elements_are_eigenvalues}
            Let $V$ be a finite-dimensional complex vector space of dimension $n = \dim V$ and consider its endomorphism algebra $\End(V)$ with the operator norm (note that $\End_{\cont}(V) = \End(V)$ since every operator on a finite-dimensional complex vector space is bounded). The spectrum of an operator $\varphi \in \End(V)$ (viewed as an $n \x n$ complex matrix) is then, by definition, the set of complex numbers $\lambda \in \bbC$ such that $\lambda I_n - \varphi$ is not invertible, where $I_n$ denotes the $n \x n$ identity matrix. We know that a finite-dimensional matrix with coefficients in a field is non-invertible if and only if its determinant is zero, so $\det(\lambda I_n - \varphi) = 0$ for all $\lambda \in \sigma(\varphi)$. Now, observe that $\det(\lambda I_n - \varphi)$ is nothing but the characteristic polynomial of $\varphi$. Therefore, complex numbers $\lambda \in \bbC$ such that $\det(\lambda I_n - \varphi) = 0$ are precisely the eigenvalues of the operator $\varphi \in \End(V)$, and hence definition \ref{def: spectra_of_operators} agrees with the usual notion of the spectrum of a finite-dimensional matrix with coefficients from a field, that being its set of eigenvalues not counting (algebraic) multiplicities.
        \end{example}
        \begin{proposition}[Spectra of non-zero operators are non-empty] \label{prop: spectra_of_non_zero_operators_are_non_empty}
            Let $A$ be a unital \textit{non-zero} Banach algebra and let $x \in A \setminus \{0\}$ be an arbitrarily chosen non-zero element. Then $\sigma(x) \not = \varnothing$. 
        \end{proposition}
            \begin{proof}
                Suppose for the sake of deriving a contradiction that there exists $x \in A \setminus \{0\}$ such that $\sigma(x) = \varnothing$. By definition \ref{def: spectra_of_operators}, this means that $\lambda - x \in A^{\x}$ for all $\lambda \in \bbC$. Next, let $\phi: A \to \bbC$ be a continuous $\bbC$-linear map\footnote{I.e. a linear functional.} and observe that $\lambda \mapsto \phi\left(\frac{1}{\lambda - x}\right)$ is a holomorphic function on $\bbC$; furtheremore, this function is bounded ($\left|\phi\left(\frac{1}{\lambda - x}\right)\right| \leq \|\phi\|_{\sup}$ for all $\lambda \in \bbC$ thanks to the assumption that $x \in A \setminus \{0\}$) on the entirety of $\bbC$ (in fact, it vanishes as $\lambda \to \infty$), which by Liouville's Theorem implies that it is actually constant. The linear functional $\phi$ is arbitrary, so the function $\lambda \mapsto \frac{1}{\lambda - x}$ is therefore constant, which in turn implies that $\lambda \mapsto \lambda - x$ is constant, i.e. $\lambda = \lambda - x$. This means that $x = 0$, but since we have assumed that $x \in A \setminus \{0\}$, it can only be the case that $A = 0$, but as we have assumed that $A$ is non-zero, this is clearly a contradiction. As such, our assumption that there exists $x \in A \setminus \{0\}$ such that $\sigma(x) = \varnothing$ is false, meaning that $\sigma(x) \not = \varnothing$ for all $x \in A \setminus \{0\}$ whenever $A \not = 0$, as claimed.
            \end{proof}
        \begin{corollary} \label{coro: banach_division_algebras_are_isomorphic_to_ground_field}
            Any complex Banach algebra $A$ that is a division algebra (hence non-zero and unital) is isomorphic as a $\bbC$-algebra to $\bbC$ itself.
        \end{corollary}
            \begin{proof}
                We know from proposition \ref{prop: spectra_of_non_zero_operators_are_non_empty} that $\sigma(x) \not = \varnothing$ for all $x \in A \setminus \{0\}$, so we can always choose $\lambda \in \sigma(x)$ for any fixed $x \in A \setminus \{0\}$. For such a $\lambda$, we have by definition \ref{def: spectra_of_operators} that $x - \lambda \not \in A^{\x}$, and because $A$ is assumed to be a division algebra, this means that $x - \lambda = 0$. But this in turn implies that $x = \lambda$, meaning that indeed $A \cong \bbC$.
            \end{proof}
        \begin{lemma}[Invertible elements of Banach algebras form open subsets] \label{lemma: invertible_elements_of_banach_algebras_form_open_subsets}
            Let $A$ be a unital Banach algebra. Then the subset $A^{\x} \subset A$ of two-sided invertible elements is open.
        \end{lemma}
            \begin{proof}
                It suffices to show that any invertible element $x \in A^{\x}$ admits an open neighbourhood $U \ni x$ consisting entirely of invertible elements; in fact, since multiplication by $x^{-1}$ is a homeomorphism for all $x \in A^{\x}$, we can simply show that the multiplicative unit $1 \in A^{\x}$ admits an open neighbourhood $U \ni 1$ of invertible elements. Actually, we might as well show that there exists a radius $\delta > 0$ so that the open $\delta$-ball $\B_{\delta}(1)$ centered at $1$ is a subset of $A^{\x}$: because $\|1 + y\| \leq 1 + \|y\|$ per the definition of norms, this amounts to showing that $1 + y \in A^{\x}$ if $\|y\|$ is sufficiently small. For this, consider firstly the fact that $1 + y = \frac{1 + y^N}{\sum_{n = 0}^N (-1)^n y^n}$, wherein $N$ is any positive integer. By taking the limit $N \to +\infty$, one see that $(1 + y)^{-1} = \sum_{n = 0}^{+\infty} (-1)^n y^n$: this is a sum that indeed converges to $1$ for all $y \in A$ such that $\|y\| < 1$, and so one can always choose an open ball $\B_{\delta}(1)$ of radius $0 < \delta < 1$ centered at $1$. As stated, this implies that $A^{\x}$ is open inside $A$.
            \end{proof}
        \begin{corollary}
            For any unital Banach algebra $A$, the subset $A^{\x} \subset A$ of two-sided invertible elements is a topological subgroup whose topology is the subspace topology inherited from $A$. 
        \end{corollary}
        \begin{remark}
            The proof strategy used for lemma \ref{lemma: invertible_elements_of_banach_algebras_form_open_subsets} also works for unital Banach algebras over non-archimedean fields\footnote{E.g. when the Banach algebra $A$ is a finite-dimensional \textit{commutative} Banach algebra over a complete non-archimedean field then $y$ can be taken to be contained in the unique maximal ideal of $A$ (for instance, one may take $A$ to be the $\Q_p$-algebra $\Q_p(p^{1/p^{\infty}})^{\wedge}$).} without any heavy modification. Instead of proving the existence of open balls of radii $0 < \delta < 1$, we observe that the sum $1 + y = \frac{1 + y^N}{\sum_{n = 0}^N (-1)^n y^n}$ converges to $1$ whenever $y$ is topologically nilpotent. 
        \end{remark}
        \begin{proposition}[Spectra are closed discs] \label{prop: spectra_of_operators_are_closed_discs}
            Let $A$ be a unital complex Banach algebra and $x \in A$ be an arbitrarily chosen element. Then $\sigma(x)$ will be a closed subset of $\bbC$ (understood to be endowed with the usual metric topology); in fact, $\sigma(x) = \{\lambda \in \bbC \mid |\lambda| \leq \|x\|\}$.
        \end{proposition}
            \begin{proof}
                $\|x\| \geq 0$ for all $x \in A$ so we might as well consider only $\lambda \not = 0$.
                
                If $\lambda \in \sigma(x)$ then $\lambda - x$ is not invertible as an element of $A$, meaning that the series\footnote{Thought of as the formal inverse of $1 - \frac{x}{\lambda}$} $\sum_{n = 0}^{+\infty} \left(\frac{x}{\lambda}\right)^n$ does not converge to $\frac{1}{1 - \frac{x}{\lambda}}$, which is the case if and only if $|\lambda| \leq \|x\|$. On the other hand, if $\lambda \in \bbC$ is such that $|\lambda| \leq \|x\|$ then again, the series $\sum_{n = 0}^{+\infty} \left(\frac{x}{\lambda}\right)^n$ will not converge to $1 - \frac{x}{\lambda}$, from which we infer that $1 - \frac{x}{\lambda}$ is not invertible; since $\lambda \not = 0$ this implies that $\lambda - x$ is also not invertible, i.e. $\lambda \in \sigma(x)$ by definition. 
            \end{proof}
        \begin{corollary}[Characters of unital complex Banach algebras are continuous] \label{coro: characters_of_unital_complex_banach_algebras_are_continuous}
            Let $A$ be a unital complex Banach algebra. Then any homomorphism $\chi: A \to \bbC$ of $\bbC$-algebras\footnote{Such homomorphisms are also called \say{\textbf{characters}}.} has sup-norm $\leq 1$ (so in particular, it is continuous).
        \end{corollary}
            \begin{proof}
                By definition $\|\chi\|_{\sup} := \sup_{x \in A} \frac{|\chi(x)|}{\|x\|}$ so showing that $\|\chi\|_{\sup} \leq 1$ amounts to showing that $|\chi(x)| \leq \|x\|$ for all $x \in A$, which is the same as showing that $\chi(x) \in \sigma(x)$ (cf. proposition \ref{prop: spectra_of_operators_are_closed_discs}). To this end, suppose for the sake of deriving a contradiction that $\chi(x) \not \in \sigma(x)$; should this be true, $\chi(x) - x$ would be invertible. However, since $\chi(\chi(x) - x) - \chi(x) - \chi(x) = 0$, which is not invertible as an element of $\bbC$, $\chi(x) - x$ must be non-invertible as an element of $A$. This contradicts the asumption that $\chi(x) \not \in \sigma(x)$, which as stated, means that indeed $\|\chi\|_{\sup} \leq 1$ as claimed.
            \end{proof}
        \begin{proposition}[Residue fields of commutative Banach algebras] \label{prop: residue_fields_of_commutative_banach_algebras}
            Let $A$ be a commutative and unital complex Banach algebra. Then, for any $\m \in \Spm A$, there is an isomorphism of $\bbC$-algebras $A/\m \cong \bbC$. 
        \end{proposition}
            \begin{proof}
                Let us firstly recall lemma \ref{lemma: invertible_elements_of_banach_algebras_form_open_subsets}, which tells us that because $1 \in A$ is invertible, it has an open neighbourhood $U$ consisting entirely of invertible elements (i.e. $U$ is a subset of $A^{\x}$). $\m$ is a proper ideal by definition and therefore does not contain any invertible element \textit{a priori}; as such, $\m \cap U = \varnothing$ for any open neighbourhood $U \ni 1$ inside $A^{\x}$. Now, the inversion map $(-)^{-1}: A \setminus \{0\} \to A$ can be easily checked to be continuous: this is a fact that we can use to show that the topological closure $\bar{\m} \supseteq \m$ also can not contain any invertible element. We can then use the additional facts that addition and scalar multiplications are also continuous maps to show that $\bar{\m}$ must therefore be a proper $A$-ideal; the maximality of $\m$ then implies that $\bar{\m} = \m$, i.e. any maximal $A$-ideal $\m \in \Spm A$ is closed as a subset of $A$. Now, endow $A/\m$ with the quotient topology induced by the canonical quotient map $A \to A/\m$ and recall that quotients of metric spaces are themselves metric spaces. It is not hard to verify that precisely because $\m$ is closed, the quotient $A/\m$ is indeed a complete metric space, and since it is already a $\bbC$-algebra via the composition, it is thus a complex Banach algebra by definition. Then, simply apply corollary \ref{coro: banach_division_algebras_are_isomorphic_to_ground_field} to get a $\bbC$-algebra isomorphism $A/\m \cong \bbC$.
            \end{proof}
            
        \begin{definition}[Spectral radii] \label{def: spectral_radii}
            Let $A$ be a unital complex Banach algebra and $x \in A$ be an arbitrarily chosen element. Its spectral radius is then defined as:
                $$\rho(x) := \sup_{\lambda \in \sigma(x)} |\lambda|$$
            Note that this number is always finite as a direct consequence of proposition \ref{prop: spectra_of_operators_are_closed_discs}.
        \end{definition}
        The following theorem is our first non-trivial result, not only because of it gives us a practical formula for computing spectral radii of elements of Banach algebras (i.e. operators) but also, because it immediately implies (cf. corollary \ref{coro: the_spectral_multiplicative_semi_norm}) that $\rho: A \to \R_{\geq 0}$ is a multiplicative semi-norm, which leads us to the theory of Berkovich spaces.  
        \begin{theorem}[Gelfand's spectral radius formula] \label{theorem: gelfand_spectral_radius_formula}
            Let $A$ be a unital complex Banach algebra and $x \in A$ be an arbitrarily chosen element. Then, the spectral radius of $x$ can be computed as:
                $$\rho(x) = \limsup_{n \to +\infty} \|x^n\|^{\frac1n} = \lim_{n \to +\infty} \|x^n\|^{\frac1n}$$
        \end{theorem}
            \begin{proof}
                
            \end{proof}
        \begin{corollary}[The spectral multiplicative semi-norm] \label{coro: the_spectral_multiplicative_semi_norm}
            The assignment of spectral radii to elements $x \in A$ of a given unital Banach algebra $(A, \|-\|)$:
                $$\rho: A \to \R_{\geq 0}$$
            is a multiplicative semi-norm\footnote{A \textbf{semi-norm} on a vector space $V$ over a normed field $(k, |\cdot|)$ is a function $\|-\|: V \to \R_{\geq 0}$ such that $\rho(\lambda v) = |\lambda| \rho(v)$ for all $\lambda \in k$ and all $v \in V$ and $\rho(v + v') \leq \rho(v) + \rho(v')$ for all $v, v' \in V$, but it is not required that $\rho(v) = 0$ implies $v = 0$. A semi-norm $\rho: A \to \R_{\geq 0}$ is \textbf{multiplicative} if $\rho|_{A \setminus \{0\}}: A \setminus \{0\} \to \R_{> 0}$ it is a homomorphism of (multiplicative) monoids, supposing that we view $\R_{> 0} \subset \R_{\geq 0}$ as a (commutative) multiplicative monoid with respect to the usual numerical multiplication.}. In fact, it is the maximal multiplicative semi-norm that is dominated by $\|-\|$.
        \end{corollary}
        
        \begin{definition}[Involutions and $\rmC^*$-algebras] \label{def: involutions}
            An \textbf{involution} on a $\bbC$-algebra $A$ (which need not be unital, commutative, nor associative) is an additive and multiplicative map (i.e. a $\Z$-algebra homomorphism) $(-)^*: A \to A$ that is \textit{idempotent} (i.e. $x^{**} = x$ for all $x \in A$) and such that $\lambda^*$ is the complex-conjugate for all $\lambda \in \bbC$ (by definition, $\bbC$ is contained inside the centre of $A$).
            
            A $\bbC$-algebra equipped with an involution is called a \textbf{$*$-algebra} or an \textbf{involutive algebra}; we prefer the latter. A \textbf{$\rmC^*$-algebra} is an involutive Banach algebra $(A, (-)^*)$ such that $\|x\|^2 = \|x^* x\|$ for all $x \in A$. Homomorphisms of involutive algebras are $\bbC$-algebra homomorphisms $f: (A, (-)^*) \to (B, (-)^{\dagger})$ such that $f(x^*) = f(x)^{\dagger}$ for all $x \in A$.
        \end{definition}
        \begin{example}[Bounded continuous functions] \label{example: continuous_functions_with_complex_conjugation}
            If $X$ is any topological space then $C^0(X)$ with pointwise complex conjugation will be an involutive algebra. If $X$ is compact then $C^0(X)$ will furthermore be a $\rmC^*$-algebra; alternatively, one might consider $C^b(X)$, which is also a $\rmC^*$-algebra.
        \end{example}
        \begin{example}[Bounded operators on Hilbert spaces]
            Let $H$ be a Hilbert space and consider the algebra $\End^b(H)$ of bounded endomorphisms on $H$. The adjunction map $(-)^{\dagger}: \End^b(H) \to \End^b(H)$ which sends bounded operators $\varphi \in \End^b(H)$ to their complex-conjugate transposition $\varphi^{\dagger}$ (after some choice of basis for $H$) can be easily checked to be involutive and as such turns $\End^b(H)$ into an involutive algebra. In fact, $(\End^b(H), (-)^{\dagger})$ is a $\rmC^*$-algebra since we have $\|\varphi^{\dagger} \varphi\| = \norm{\varphi}^2$.
        \end{example}
        \begin{example}[$L^1$-functions on unimodular groups] 
            A topological group $G$ whose Haar measure $\mu$ is right-invariant via $g \cdot \mu = \mu \cdot g^{-1}$ in addition to being left-invariant is said to be \textbf{unimodular}. Then, by equipping the $\bbC$-vector space $L^1(G)$ of absolutely integrable functions on $G$ can be endowed with convolution and the involution $(-)^{\dagger}: L^1(G) \to L^1(G)$ determined via $f^{\dagger}(g) := f(g^{-1})^*$ for all $g \in G$ to turn it into an involutive algebra. However, this is generally not a $C^*$-algebra, since \textit{a priori} do not have $\norm{f}_{L^1(G)}^2 = \|f^{\dagger} f\|_{L^1(G)}$.
        \end{example}
        \begin{convention}
            There is a category of involutive algebras, denoted by $*\-\Assoc\Alg$, wherein involutive algebras are the objects and homomorphisms between them are the morphisms. $\rmC^*$-algebras form a full subcategory of this category\footnote{Check this!} which shall be denoted by $\rmC^*\-\Assoc\Alg$. 
        \end{convention}
            
        \begin{definition}[Normal elements] \label{def: normal_elements}
            An element $x \in A$ of an involutive algebra $(A, (-)^*)$ is called \textbf{normal} if and only if $[x, x^*] = 0$.
        \end{definition}
        The following lemma will be used in the proof of theorem \ref{theorem: gelfand_duality}.
        \begin{lemma}[Spectral radii and norms of normal elements] \label{prop: spectral_radii_and_norms_of_normal_elements}
            Let $(A, \|-\|, (-)^*)$ be a $\rmC^*$-algebra and $x \in A$ be a fixed normal element. Then $\rho(x) = \|x\|$.
        \end{lemma}
            \begin{proof}
                Combine theorem \ref{theorem: gelfand_spectral_radius_formula} with definitions \ref{def: involutions} and \ref{def: normal_elements}.
            \end{proof}
            
        \begin{definition}[Gelfand spectra] \label{def: gelfand_spectra}
            The \textbf{Gelfand spectrum} of a unital complex Banch algebra $A$ is given by:
                $$\rmC^*\-\Spec A := \bbC\-\Alg(A, \bbC)$$
            where $\bbC\-\Alg$ denotes the category where the objects are $\bbC$-algebras, which need not be associative, comnmutative, nor even unital, and the morphisms are $\bbC$-algebra homomorphisms. We might also refer to this as the \textbf{$\rmC^*$-spectrum} of $A$, for reasons that will become clear after theorem \ref{theorem: gelfand_duality}.
        \end{definition}
        \begin{remark}
            Note that all $\bbC$-algebra homomorphism $\chi \in \bbC\-\Alg(A, \bbC)$ are continuous by corollary \ref{coro: characters_of_unital_complex_banach_algebras_are_continuous}. Furthermore, the assignment $\rmC^*\-\Spec: \bbC\Ban\Assoc\Alg^{\op} \to \Sets$ is a functor; in fact, this extends to a functor $\rmC^*\-\Spec: \bbC\Ban\Assoc\Alg^{\op} \to \Top$ since the set $\rmC^*\-\Spec A$ can be endowed with the sup-norm to get turned into a metric space. 
        \end{remark}
        \begin{theorem}[Gelfand Duality] \label{theorem: gelfand_duality}
            \noindent
            \begin{enumerate}
                \item \textbf{(Unital Gelfand Duality):} There is an adjoint equivalence between the opposite $\rmC^*\-\Comm\Alg^{\op}$ of the category of commutative $\rmC^*$-algebras and the category $\Comp$ of compacta\footnote{Plural of \say{compactum}. A \textbf{compactum} is a compact and Hausdorff topological space.} as follows:
                    $$
                        \begin{tikzcd}
                            {\rmC^*\-\Comm\Alg^{\op}} & \Comp
                            \arrow[""{name=0, anchor=center, inner sep=0}, "{\rmC^*\-\Spec}"', bend right, from=1-1, to=1-2]
                            \arrow[""{name=1, anchor=center, inner sep=0}, "{C^0(-)}"', bend right, from=1-2, to=1-1]
                            \arrow["\dashv"{anchor=center, rotate=-90}, draw=none, from=1, to=0]
                        \end{tikzcd}
                    $$
                \item \textbf{(Non-unital Gelfand Duality):} There is an adjoint equivalence between the opposite $\rmC^*\-\Semi\Comm\Alg^{\op}$ of the category of not-necessarily unital commutative $\rmC^*$-algebras and the category ${}^{\pt/}\Comp$ of pointed compacta as follows:
                    $$
                        \begin{tikzcd}
                            {\rmC^*\-\Semi\Comm\Alg^{\op}} & { \Comp}
                            \arrow[""{name=0, anchor=center, inner sep=0}, "{\rmC^*\-\Spec}"', bend right, from=1-1, to=1-2]
                            \arrow[""{name=1, anchor=center, inner sep=0}, "{C^0(-)}"', bend right, from=1-2, to=1-1]
                            \arrow["\dashv"{anchor=center, rotate=-90}, draw=none, from=1, to=0]
                        \end{tikzcd}
                    $$
                wherein the functor $C^0(-): {}^{\pt/}\Comp \to \rmC^*\-\Semi\Comm\Alg^{\op}$ is understood to be given by $C^0(X, x_0) := \{f \in C^0(X) \mid f(x_0) = 0\}$.
            \end{enumerate}
        \end{theorem}
            \begin{proof}
                \noindent
                \begin{enumerate}
                    \item \textbf{(Unital Gelfand Duality):} 
                    \item \textbf{(Non-unital Gelfand Duality):} 
                \end{enumerate}
            \end{proof}
        
        \begin{definition}[Hermitian and skew-symmetric elements] \label{def: hermitian_and_skew_symmetric_elements}
            Following terminologies from elementary linear algebra, we say that an element $x \in A$ of an involutive algebra $(A, (-)^*)$ is \textbf{Hermitian} if and only if $x^* = x$,  \textbf{skew-symmetric} if and only if $x^* = -x$.
        \end{definition}
        \begin{definition}[Positivity and negativity] \label{def: positivity_and_negativity}
            Let $(A, (-)^*)$ be an involutive algebra. A Hermitian element $x \in A$ is \textbf{positive} (respectively, \textbf{negative}) if and only if $\sigma(x) \subseteq \R_{\geq 0}$ (respectively, $\sigma(x) \subseteq \R_{\leq 0}$). An element $x \in A$ is real if and only if $\sigma(x) \subseteq \R$.
        \end{definition}
        \begin{proposition}[Hermitian elements are real] \label{prop: hermitian_elements_are_real}
            Hermitian elements in any involutive algebra are always real.
        \end{proposition}
            \begin{proof}
                
            \end{proof}
        \begin{example}
            Let $X$ be a topological space and consider the involutive algebra $(C^0(X), (-)^*)$ (cf. example \ref{example: continuous_functions_with_complex_conjugation}). Then, the non-negative functions $f: X \to \R_{\geq 0}$ are evidently are the positive elements.
        \end{example}
            
        \begin{remark}[Stone-\v{C}ech Compactifications]
            
        \end{remark}

\subsection{von Neumann algebras}