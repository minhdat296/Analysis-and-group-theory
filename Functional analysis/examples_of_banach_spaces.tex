\section{Examples of Banach and Hilbert spaces}
    \subsection{\texorpdfstring{$L^p$}{}-spaces} \label{subsection: L_p_spaces}
        \begin{convention}
            Let $(X, \mu)$ be a measure space. Unless specified to be otherwise, we shall be writing:
                $$L^p(X, \mu) := \left\{ f: X \to K \bigg\mid \norm{f}_{L^p(X, \mu)} := \left( \int_X \abs{f}^p d\mu \right)^{\frac1p} < +\infty \right\}$$
            for all $p \in \N$.

            For $p = \infty$, let us say that a function $f: X \to K$ is essentially bounded by some constant $M \in \R_{\geq 0}$ if and only if $\abs{f(x)} \leq M$ for $\mu$-almost every $x \in X$. For such functions $f$, we can define the following quantity:
                $$\norm{f}_{L^{\infty}(X, \mu)} := \inf\{ M \in \R_{\geq 0} \mid \abs{f(x)} \leq M \}$$
            We then define:
                $$L^{\infty}(X, \mu) := \left\{ f: X \to K \mid \norm{f}_{L^{\infty}(X, \mu)} < +\infty \right\}$$
        \end{convention}
    
        For what follows, recall that in a vector space $E$, a subset $K \subseteq E$ is said to be \textbf{convex} if and only if for every $x, y \in E$ and every $\lambda \in [0, 1]$, we have that:
            $$\lambda(x - y) + y \in K$$
        and a function $J: K \to \R$ is said to be \textbf{convex} if and only if:
            $$J( \lambda(x - y) + y ) \leq \lambda( J(x) - J(y) ) + J(y)$$
        for all $\lambda \in [0, 1]$. What the first definition is saying is that a subset $K$ of a vector space is convex if and only if given any $x, y \in K$, the line segment between those two points must also lie entirely within $K$; notice that points on said line segment are precisely given by:
            $$x_{\lambda} = \lambda(x - y) + y$$
        each for a specified $\lambda \in [0, 1]$, and as the parameter $\lambda$ varies, we travel from $x$ to $y$ (or vice versa). The second definition therefore is saying that a function is convex if and only if it preserves this property whereby line segments remain inside.

        It is also worth noting that convex sets are, by definition (since we are considering line segments in the definition), path-connected. Consequently, they are connected. 
        
        Also, if $(X, \mu)$ is any measure space (we are suppressing the $\sigma$-algebra) and $f: X \to \R$ is a measurable function, then let us write:
            $$\bbE[f] := \frac{1}{\mu(X)} \int_X f d\mu$$
        This is the \textbf{average value} of $f$ (or the \textbf{expected value} if the domain of $f$ is $[0, 1]$ and if $\mu(X) = 1$, because in that case $\mu$ would be a probability measure).
        \begin{lemma}[Convex functions are continuous]
            If $E$ is a finite-dimensional normed vector space and $K$ is a convex subset thereof, then any convex function $J: K \to \R$ will be continuous. 
        \end{lemma}
            \begin{proof}
                First of all, we claim that it is enough to show that any convex function $J: (a, b) \to \R$ is necessarily continuous. 
            \end{proof}
        \begin{lemma}[Jensen's inequality] \label{lemma: jensen_inequality}
            Let $(X, \mu)$ be a measure space and $J: \R \to \R$ be a convex function. Then, we shall have that:
                $$J( \bbE[f] ) \leq \bbE[ J \circ f ]$$
            for any real-valued $f \in L^1(X, \mu)$. 
        \end{lemma}
            \begin{proof}
                
            \end{proof}
        \begin{lemma}[H\"older's inequality]
            Let $p, q \in \N_{\geq 1} \cup \{+\infty\}$ be such that:
                $$\frac1p + \frac1q = 1$$
            Let $(X, \mu)$ be a measure space and let $f \in L^p(X, \mu), g \in L^q(X, \mu)$. Then, we have the following inequality:
                $$\norm{f}_{L^p(X, \mu)} \norm{g}_{L^q(X, \mu)} \geq \norm{fg}_{L^1(X, \mu)}$$
            and hence:
                $$fg \in L^1(X, \mu)$$
        \end{lemma}
            \begin{proof}
                \todo[inline]{Apply Radon-Nikodym to find a new probability measure $\nu$ on $X$ (what do we use as the Radon-Nikodym derivative ?) for which we can apply Jensen's inequality to the convex function $\norm{-}^p$ (this is why $p \geq 0$ is crucial).}
            \end{proof}
        \begin{corollary}[$L^p$-spaces inclusions]
            If $(X, \mu)$ is a measure space (i.e. $\mu(X) < +\infty$) then:
                $$\forall p, q \in \N_{\geq 1} \cup \{+\infty\}: p \leq q \implies L^p(X, \mu) \subset L^q(X, \mu)$$
            if and only if $X$ is $\mu$-finite, i.e.:
                $$\mu(X) < +\infty$$ 
        \end{corollary}
            \begin{proof}
                
            \end{proof}

        \begin{lemma}[Minkowski's inequality] \label{lemma: minkowski_inequality}
            For any measure space $(X, \mu)$, the pairs:
                $$(L^p(X, \mu), \norm{-}_{L^p(X, \mu)})$$
            are normed spaces.
        \end{lemma}
            \begin{proof}
                \todo[inline]{This relies on H\"older's inequality.}
            \end{proof}
        \begin{theorem}[$L^p$-space duality] \label{theorem: L_p_space_duality}
            Let $p, q \in \N_{\geq 1} \cup \{+\infty\}$ be such that:
                $$\frac1p + \frac1q = 1$$
            Let $(X, \mu)$ be a measure space. Then, there shall be homeomorphisms:
                $$L^q(X, \mu) \cong L^p(X, \mu)^*_{\weak}$$
        \end{theorem}
            \begin{proof}
                
            \end{proof}
        \begin{remark}
            The homeomorphisms $L^q(X, \mu) \cong L^p(X, \mu)^*_{\weak}$ are generally not isometric, since in general, the weak topology is not even a metric topology to begin with.
        \end{remark}
        \begin{theorem}[$L^p$-spaces are complete] \label{theorem: L_p_space_completeness}
            The normed spaces $(L^p(X, \mu), \norm{-}_{L^p(X, \mu)})$ are complete with respect to their metric topologies.
        \end{theorem}
            \begin{proof}
                \todo[inline]{Minkowski's inequality + some standard $\e$-$\delta$ convergence arguments.}
            \end{proof}

        \todo[inline]{$L^p$-spaces (infinite measures)}

        \begin{lemma}[Comparing $L^p$ and $L^q$-norms: finite-measure cases] \label{lemma: comparing_L_p_norms_for_finite_measures}
            Let $(X, \Sigma, \mu)$ be a measure space and choose $1 \leq p, q \leq +\infty$. If $p < q$ and $\mu(X)$ is finite, then:
                $$\norm{f}_{L^p(X, \mu)} \leq \mu(X)^{\frac1p - \frac1q} \norm{f}_{L^q(X, \mu)}$$
            Furthermore, the constant $\mu(X)^{\frac1p - \frac1q}$ is sharp.
        \end{lemma}
            \begin{proof}
                Let us apply H\"older's inequality to the pair of functions $1$ and $f^p$, using the pair of conjugate exponents $\frac{1}{1 - \left(\frac{q}{p}\right)^{-1}} = \frac{1}{1 - \frac{p}{q}} = \frac{q}{q - p}$ and $\frac{q}{p}$ (note that since $p < q$ by assumption, we indeed have that $\frac{q}{p} \geq 1$), which yields:
                    $$\norm{f}_{L^p(X, \mu)}^p = \int_X \abs{f}^p d\mu = \norm{ 1 \cdot f^p }_{L^1(X, \mu)} \leq \norm{1}_{L^{\frac{q}{q - p}}(X, \mu)} \norm{f^p}_{L^{\frac{q}{p}}(X, \mu)}$$
                Now, we have:
                    $$\norm{1}_{L^{\frac{q}{q - p}}(X, \mu)} = \left(\int_X \abs{1}^{\frac{q}{q - p}} d\mu\right)^{ \frac{1}{\frac{q}{q - p}} } = \mu(X)^{ \frac{q - p}{q} }$$
                    $$\norm{f^p}_{L^{\frac{q}{p}}(X, \mu)} = \left( \int_X \abs{f}^q d\mu \right)^{\frac{p}{q}} = \norm{f}_{L^q(X, \mu)}^p$$
                Lastly, since $p \geq 1$, the function $(-)^{\frac1p}: \R_{\geq 0} \to \R_{\geq 0}$ is monotonic. By putting everything together, we yield:
                    $$\norm{f}_{L^p(X, \mu)} \leq \mu(X)^{ \frac{q - p}{qp} } \norm{f}_{L^q(X, \mu)} = \mu(X)^{\frac1p - \frac1q} \norm{f}_{L^q(X, \mu)}$$
                which is as desired.
            \end{proof}
            
        \begin{example}[$\ell^p$-spaces] \label{example: ell_p_spaces}
            The datum of a $K$-valued function on the finite set $\{1, ..., n\}$ is the same as a vector in $K^{\oplus n}$. If we endow $\{1, ..., n\}$ with the $\sigma$-algebra $\calP(\{1, ..., n\})$ (the power set of $X$) and with the counting measure:
                $$\mu_{\discrete}: \calP(\{1, ..., n\}) \to \R_{\geq 0}$$
            which is given by:
                $$\mu_{\discrete}(S) := |S|$$
            for all $S \in \calP(\{1, ..., n\})$ (and hence $\{1, ..., n\}$ is a measure-finite space), then the normed spaces:
                $$\ell^p(K^{\oplus n}) := (K^{\oplus n}, \norm{-}_{L^p(\{1, ..., n\}, \mu)})$$
            and:
                $$L^p(\{1, ..., n\}, \mu_{\discrete}\})$$
            for every $p \in \N_{\geq 1} \cup \{+\infty\}$ will be isomorphic to one another. To be thorough, let us note that in this case, the norm $\norm{-}_{L^p(\{1, ..., n\}, \mu)}$ is given by:
                $$\norm{(x_1, ..., x_n)}_{L^p(X, \mu)} := \left( \int_{\{1, ..., n\}} |x|^p d\mu_{\discrete} \right)^{\frac1p} = \left( \sum_{i = 1}^n |x_i|^p \right)^{\frac1p}$$
            for all functions $x: \{1, ..., n\} \to K$ given by $x(i) := x_i$ for all $1 \leq i \leq n$. $\ell^p(K^{\oplus n})$ is therefore an example of a finite-dimensional $L^p$-space.

            For each $p, q \in \N_{\geq 1} \cup \{+\infty\}$ such that $\frac1p + \frac1q = 1$, we have a homeomorphism:
                $$\ell^q(K^{\oplus n}) \cong \ell^p(K^{\oplus n})^*_{\weak}$$
            by theorem \ref{theorem: L_p_space_duality}, and by finite-dimensionality, the homeomorphism above is actually an isometry (see theorem \ref{theorem: finite_dimensional_weak_duality}). Also by finite-dimensionality, each $\varphi \in \ell^p(K^{\oplus n})^*_{\weak}$ is representable by some $y := (y_1, ..., y_n) \in \ell^q(K^{\oplus n})$ in the sense that:
                $$\varphi(x) := x \cdot y := \sum_{i = 1}^n x_i y_i$$
            for all $x := (x_1, ..., x_n) \in \ell^p(K^{\oplus n})$, where $\cdot$ is the usual dot product. The discrete version of H\"older's inequality thus reads:
                $$\abs{x \cdot y} \leq \norm{x}_{L^p(X, \mu)} \norm{y}_{L^q(X, \mu)}$$

            Each $\ell^p(K^{\oplus n})$ is also complete with respect to the topology defined by its norm. Note also that when $p = q = 2$, $\ell^2(K^{\oplus n})$ will be a Hilbert space.
        \end{example}

        \begin{convention}
            For every $1 \leq p \leq +\infty$ and every set $\Gamma$, let us abbreviate:
                $$\ell^p(\Gamma) := \ell^p(K^{\oplus \Gamma})$$
            and:
                $$\norm{-}_p := \norm{-}_{\ell^p(K^{\oplus \Gamma})}$$
        \end{convention}

        Now, even though norms on finite-dimensional vector spaces in general are all equivalent to one another \textit{a priori}\todo{Write a lemma for this fact.}, in the case of $\ell^p$-norms on finite-dimensional vector spaces, we can actually write down explicitly how much they differ from one another.
        \begin{lemma}[Sharp comparison of $\ell^p$-norms in finite-dimensions] \label{lemma: sharp_comparison_of_ell_p_norms_in_finite_dimensions}
            Fix $n \in \N$ and let $1 \leq p < q < +\infty$. Then:
                $$c \norm{-}_p \leq \norm{-}_q \leq C \norm{-}_p$$
            as norms on $K^{\oplus n}$, wherein $c = n^{\frac1q - \frac1p}$ and $C = 1$. Furthermore, these constants are sharp.
        \end{lemma}
            \begin{proof}
                Since $\mu_{\discrete}(\{1, ..., n\}) = n$ is finite by construction, lemma \ref{lemma: comparing_L_p_norms_for_finite_measures} applies and tells us that if $p < q$ then:
                    $$\norm{-}_p \leq n^{\frac1p - \frac1q} \norm{-}_q$$
                and that the constant $n^{\frac1p - \frac1q}$ is sharp, so we can take:
                    $$c := n^{-(\frac1p - \frac1q)} = $$

                To find $C$, let us use to fact that norms homogeneous to assume without any loss of generality that $\norm{x}_q = 1$. This forces $\abs{x_i}^q \leq 1$ for all $1 \leq i \leq n$, and hence $p < q$ implies that:
                    $$\norm{x}_q = \left( \int_{\{1, ..., n\}} \abs{x}^q d\mu_{\discrete} \right)^{\frac1q} \leq \left( \int_{\{1, ..., n\}} \abs{x}^p d\mu_{\discrete} \right)^{\frac1p} = \norm{x}_p$$
                Thus, we can take:
                    $$C := 1$$
            \end{proof}

        \todo[inline]{$L^p$-spaces (infinite measures)}

         \begin{lemma}[$L^p$-space inclusions for infinite measures] \label{lemma: L_p_space_inclusions_infinite_measures}
            Let $(X, \Sigma, \mu)$ be a measure space and choose $1 \leq p, q \leq +\infty$. If $p < q$ and:
                $$\inf_{S \in \Sigma, S \not = \varnothing} \mu(S) > 0$$
            (i.e. there are no non-empty measure-zero sets), and:
                $$\mu(S) = +\infty$$
            then there will be a strict inclusion:
                $$L^p(X, \mu) \subset L^q(X, \mu)$$
        \end{lemma}
            \begin{proof}
                For each $n \in \N$, let:
                    $$X_{> n} := \{ x \in X \mid \forall f \in L^p(X, \mu): \abs{f(x)} > n \}$$
                Then, we have that:
                    $$\norm{f}_{L^p(X, \mu)}^p := \int_X \abs{f}^p d\mu \geq \sum_{n \geq 0} \int_{X_{> n}} \abs{ n \chi_{X_{> n}} }^p d\mu = \sum_{n \geq 0} \int_{X_{> n}} n^p d\mu = \sum_{n \geq 0} n^p \mu(X_{> n})$$
                From this, we gather that:
                    $$\forall n \in \N: \frac{1}{n^p} \norm{f}_{L^p(X, \mu)} \geq \mu(X_{> n})$$
                and hence:
                    $$n \gg 0 \implies 0 \geq \mu(X_{> n})$$
                and since $\mu$ takes values in $\R_{\geq 0}$, the above implies that $\mu(X_{> n}) = 0$ whenever $n \gg 0$. As we have assumed that:
                    $$\inf_{S \in \Sigma, S \not = \varnothing} \mu(S) > 0$$
                i.e.:
                    $$\mu(S) = 0 \iff S = \varnothing$$
                the fact that $\mu(X_{> n}) = 0$ whenever $n \gg 0$ now implies that:
                    $$n \gg 0 \implies X_{> n} = \varnothing$$
                This means that for $n \gg 0$, there are no functions $f \in L^p(X, \mu)$ for which there exists any $x\ in X$ such that $\abs{f(x)} > n$; in other words, any $f \in L^p(X, \mu)$ must be bounded, i.e. $\norm{f}_{\infty}$ is finite.

                Now, observe that since $1 \leq p < q$, the functions $(-)^p, (-)^q: \R_{\geq 0} \to \R_{\geq 0}$ are monotonic and we also have that:
                    $$(-)^q \leq (-)^p$$
                Next, let:
                    $$X_{\leq 1} := \{ x \in X \mid \forall f \in L^p(X, \mu): \abs{f(x)} \leq 1 \}$$
                and note that:
                    $$X = X_{\leq 1} \sqcup X_{> 1}$$
                which means that:
                    $$\int_X (-) d\mu = \int_{X_{\leq 1}} (-) d\mu + \int_{X_{> 1}} (-) d\mu$$
                We are now led to consider the following:
                    $$\norm{f}_{L^q(X, \mu)}^q := \int_X \abs{f}^q d\mu = \int_{X_{\leq 1}} \abs{f}^q d\mu + \int_{X_{> 0}} \abs{f}^q d\mu \leq \int_{X_{\leq 1}} \abs{f}^p d\mu + \norm{f}_{\infty} = \norm{f}_{L^p(X_{> 1}, \mu)}^p + \norm{f}_{\infty}$$
                Since $f \in L^p(X, \mu)$ and since $X_{> 1}$ is a measurable subset of $X$ (and also because $f$ is bounded \textit{a priori}), the expression $\norm{f}_{L^p(X_{> 1}, \mu)}^p + \norm{f}_{\infty}$ is finite, and hence we can conclude that $f \in L^q(X, \mu)$. We thus have:
                    $$L^p(X, \mu) \subset L^q(X, \mu)$$
                and the inclusion is strict because $\abs{f}^q = \abs{f}^p$ if and only if $\abs{f} = 1$ (supposing that $p < q$).
            \end{proof}

        \begin{example}[Separability of $\ell^p$ of countable dimensions (finite $p$)] \label{example: ell_p_spaces_of_countable_dimensions_are_separable}
            Let $\Gamma$ be a set and let $1 \leq p < +\infty$. If $\Gamma$ is countable then $\ell^p(\Gamma)$ will be separable.

            \todo[inline]{Add a proof}
        \end{example}
        \begin{example}[Inseparability of $\ell^p$ of uncountable dimensions (finite $p$)] \label{example: ell_p_spaces_of_uncountable_dimensions_finite_p_are_inseparable}
            Let $\Gamma$ be a set and let $1 \leq p < +\infty$. If $\Gamma$ is uncountable then $\ell^p(\Gamma)$ will be inseparable.
        \end{example}
        \begin{example}[Inseparability of $\ell^{\infty}$ of infinite dimensions] \label{example: ell_infinity_spaces_of_infinite_dimensions_is_inseparable}
            Let $\Gamma$ be an infinite set. We claim that:
                $$c_0(\Gamma) := \{ f: \Gamma \to \bbK \mid \forall \e > 0: \text{the set $\{ \gamma \in \Gamma \mid \abs{f(\gamma)} > \e \}$ is finite} \}$$
            is an inseparable, and hence:
                $$\ell^{\infty}(\Gamma)$$
            will also be inseparable, as it contains $c_0(\Gamma)$ as a normed subspace.

            We will prove the main claim regarding the inseparability of $c_0(\Gamma)$ and hence of $\ell^{\infty}(\Gamma)$ later. First of all, let us justify that the former is a normed subspace of the latter. 
        
            Consider $X := c_0(\Gamma)$. Any vector subspace of a normed space is also a normed space with respect to the subspace topology, via restricting the domain of the norm, so it is sufficient to show that $c_0(\Gamma)$ is a vector subspace of $\ell^{\infty}(\Gamma)$.
                    
            For this, recall firstly that $\ell^{\infty}(\Gamma) = L^{\infty}(\Gamma, \mu)$ consists of functions $f: \Gamma \to \bbK$ which are $\mu$-integrable and bounded $\mu$-almost everywhere. By construction, $\mu(S) \not = 0$ for all countable subsets $S \subset \Gamma$, so the unbounded locus of $f$ must either be empty, when $\Gamma$ is countable, or an uncountable subset of $\Gamma$ when $\Gamma$ is uncountable. One sees thus that $\ell^{\infty}(\Gamma)$ coincides with the space of functions $f: \Gamma \to \bbK$ such that $f|_S$ is bounded for any countable subset $S \subset \Gamma$.
            
            Next, note that the sets $\{ \gamma \in \Gamma \mid \abs{f(\gamma)} > \e \}$ being finite for all $\e > 0$ implies that if $f \in c_0(\Gamma)$ then $f(\gamma) = 0$ for all but finitely many $\gamma \in \Gamma$. Therefore, any $f \in c_0(\Gamma)$ will automatically be bounded after restricted to some countable subset of $\Gamma$, and hence $c_0(\Gamma)$ is a subset of $\ell^{\infty}(\Gamma)$. It is clear that if $f, g \in c_0(\Gamma)$ and $a, b \in \bbK$ then we will also have that $af(\gamma) + bg(\gamma) = 0$ for all but finitely many $\gamma \in \Gamma$, and hence $c_0(\Gamma)$ is a vector subspace of $\ell^{\infty}(\Gamma)$. The pair $( c_0(\Gamma), \norm{-}_{\infty} )$ is therefore a normed space.
        \end{example}
        \begin{proof}[Proof of claims in examples \ref{example: ell_p_spaces_of_uncountable_dimensions_finite_p_are_inseparable} and \ref{example: ell_infinity_spaces_of_infinite_dimensions_is_inseparable}]
            Let us now give a uniform argument for the claims in examples \ref{example: ell_p_spaces_of_uncountable_dimensions_finite_p_are_inseparable} and \ref{example: ell_infinity_spaces_of_infinite_dimensions_is_inseparable}. Hence forth, let $X$ be:
            \begin{itemize}
                \item either $\ell^p(\Gamma)$ for $\Gamma$ uncountable and $1 \leq p < +\infty$,
                \item or $\ell^{\infty}(\Gamma)$ for any infinite set $\Gamma$, countable or not.
            \end{itemize}
            The claim is that $X$ is \underline{inseparable} as a topological space.

            \todo[inline]{Add a proof}
        \end{proof}

        \todo[inline]{$p$-direct sums}

        \begin{convention}
            If $V, W$ are normed spaces and $1 \leq p \leq +\infty$, then it is typical to write:
                $$V \oplus_p W$$
            to mean $V \oplus W$ (the usual algebraic direct sum, \textit{sans topologie}) equipped with the norm given for all $(v, w) \in V \oplus W$ by:
                $$
                    \norm{(v, w)}_{V \oplus_p W} :=
                    \begin{cases}
                        \text{$(\norm{v}_V^p + \norm{w}_W^p)^{\frac1p}$ if $1 \leq p < +\infty$}
                        \\
                        \text{$\max\{ \norm{v}_V, \norm{w}_W \}$ if $p = +\infty$}
                    \end{cases}
                $$
            (cf. \cite[Example, p. 24]{litvak_functional_analysis_notes}). Let us refer to $V \oplus_p W$ as the $p$-direct sum of $V$ and $W$.
        \end{convention}

    \subsection{Existence of uniqueness of solutions to ODEs}

    \subsection{Sobolev spaces}