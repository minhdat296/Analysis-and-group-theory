\section{Examples of Banach and Hilbert spaces}
    \subsection{\texorpdfstring{$L^p$}{}-spaces}
        \begin{convention}
            Let $(X, \mu)$ be a measure space. Unless specified to be otherwise, we shall be writing:
                $$L^p(X, \mu) := \left\{ f: X \to K \bigg\mid \norm{f}_{L^p(X, \mu)} := \left( \int_X \abs{f}^p d\mu \right)^{\frac1p} < +\infty \right\}$$
            for all $p \in \N$.

            For $p = \infty$, let us say that a function $f: X \to K$ is essentially bounded by some constant $M \in \R_{\geq 0}$ if and only if $\abs{f(x)} \leq M$ for $\mu$-almost every $x \in X$. For such functions $f$, we can define the following quantity:
                $$\norm{f}_{L^{\infty}(X, \mu)} := \inf\{ M \in \R_{\geq 0} \mid \abs{f(x)} \leq M \}$$
            We then define:
                $$L^{\infty}(X, \mu) := \left\{ f: X \to K \mid \norm{f}_{L^{\infty}(X, \mu)} < +\infty \right\}$$
        \end{convention}
    
        For what follows, recall that in a vector space $E$, a subset $K \subseteq E$ is said to be \textbf{convex} if and only if for every $x, y \in E$ and every $\lambda \in [0, 1]$, we have that:
            $$\lambda(x - y) + y \in K$$
        and a function $J: K \to \R$ is said to be \textbf{convex} if and only if:
            $$J( \lambda(x - y) + y ) \leq \lambda( J(x) - J(y) ) + J(y)$$
        for all $\lambda \in [0, 1]$. What the first definition is saying is that a subset $K$ of a vector space is convex if and only if given any $x, y \in K$, the line segment between those two points must also lie entirely within $K$; notice that points on said line segment are precisely given by:
            $$x_{\lambda} = \lambda(x - y) + y$$
        each for a specified $\lambda \in [0, 1]$, and as the parameter $\lambda$ varies, we travel from $x$ to $y$ (or vice versa). The second definition therefore is saying that a function is convex if and only if it preserves this property whereby line segments remain inside.

        It is also worth noting that convex sets are, by definition (since we are considering line segments in the definition), path-connected. Consequently, they are connected. 
        
        Also, if $(X, \mu)$ is any measure space (we are suppressing the $\sigma$-algebra) and $f: X \to \R$ is a measurable function, then let us write:
            $$\bbE[f] := \frac{1}{\mu(X)} \int_X f d\mu$$
        This is the \textbf{average value} of $f$ (or the \textbf{expected value} if the domain of $f$ is $[0, 1]$ and if $\mu(X) = 1$, because in that case $\mu$ would be a probability measure).
        \begin{lemma}[Convex functions are continuous]
            If $E$ is a finite-dimensional normed vector space and $K$ is a convex subset thereof, then any convex function $J: K \to \R$ will be continuous. 
        \end{lemma}
            \begin{proof}
                First of all, we claim that it is enough to show that any convex function $J: (a, b) \to \R$ is necessarily continuous. 
            \end{proof}
        \begin{lemma}[Jensen's inequality] \label{lemma: jensen_inequality}
            Let $(X, \mu)$ be a measure space and $J: \R \to \R$ be a convex function. Then, we shall have that:
                $$J( \bbE[f] ) \leq \bbE[ J \circ f ]$$
            for any real-valued $f \in L^1(X, \mu)$. 
        \end{lemma}
            \begin{proof}
                
            \end{proof}
        \begin{lemma}[H\"older's inequality]
            Let $p, q \in \N_{\geq 1} \cup \{+\infty\}$ be such that:
                $$\frac1p + \frac1q = 1$$
            Let $(X, \mu)$ be a measure space and let $f \in L^p(X, \mu), g \in L^q(X, \mu)$. Then, we have the following inequality:
                $$\norm{f}_{L^p(X, \mu)} \norm{g}_{L^q(X, \mu)} \geq \norm{fg}_{L^1(X, \mu)}$$
            and hence:
                $$fg \in L^1(X, \mu)$$
        \end{lemma}
            \begin{proof}
                \todo[inline]{Apply Radon-Nikodym to find a new probability measure $\nu$ on $X$ (what do we use as the Radon-Nikodym derivative ?) for which we can apply Jensen's inequality to the convex function $\norm{-}^p$ (this is why $p \geq 0$ is crucial).}
            \end{proof}
        \begin{corollary}[$L^p$-spaces inclusions]
            If $(X, \mu)$ is a measure space (i.e. $\mu(X) < +\infty$) then:
                $$\forall p, q \in \N_{\geq 1} \cup \{+\infty\}: p \leq q \implies L^p(X, \mu) \subset L^q(X, \mu)$$
            if and only if $X$ is $\mu$-finite, i.e.:
                $$\mu(X) < +\infty$$ 
        \end{corollary}
            \begin{proof}
                
            \end{proof}

        \begin{lemma}[Minkowski's inequality] \label{lemma: minkowski_inequality}
            For any measure space $(X, \mu)$, the pairs:
                $$(L^p(X, \mu), \norm{-}_{L^p(X, \mu)})$$
            are normed spaces.
        \end{lemma}
            \begin{proof}
                \todo[inline]{This relies on H\"older's inequality.}
            \end{proof}
        \begin{theorem}[$L^p$-space duality] \label{theorem: L_p_space_duality}
            Let $p, q \in \N_{\geq 1} \cup \{+\infty\}$ be such that:
                $$\frac1p + \frac1q = 1$$
            Let $(X, \mu)$ be a measure space. Then:
                $$L^q(X, \mu) \cong L^p(X, \mu)^*_{\cont}$$
        \end{theorem}
            \begin{proof}
                
            \end{proof}
        \begin{theorem}[$L^p$-spaces are complete] \label{theorem: L_p_space_completeness}
            The normed spaces $(L^p(X, \mu), \norm{-}_{L^p(X, \mu)})$ are complete with respect to their metric topologies.
        \end{theorem}
            \begin{proof}
                \todo[inline]{Minkowski's inequality + some standard $\e$-$\delta$ convergence arguments.}
            \end{proof}
        \begin{example}[$\ell^p$-spaces]
            The datum of a $K$-valued function on the finite set $\{1, ..., n\}$ is the same as a vector in $K^n$. If we endow $X$ with the $\sigma$-algebra $\calP(X)$ (the power set of $X$) and with the counting measure:
                $$\mu_{\discrete}: \calP(X) \to \R_{\geq 0}$$
            which is given by:
                $$\mu_{\discrete}(S) := |S|$$
            (and hence $\{1, ..., n\}$ is a measure-finite space) then the normed spaces $\ell^p(K^n) := (K^n, \norm{-}_{L^p(X, \mu)})$ and $L^p(\{1, ..., n\}, \mu_{\discrete}\})$ for every $p \in \N_{\geq 1} \cup \{+\infty\}$ will be isometric to one another; to be thorough, let us note that in this case, the norm $\norm{-}_{L^p(X, \mu)}$ is given by:
                $$\norm{(x_1, ..., x_n)}_{L^p(X, \mu)} := \left( \int_{\{1, ..., n\}} |x|^p d\mu_{\discrete} \right)^{\frac1p} = \left( \sum_{i = 1}^n |x_i|^p \right)^{\frac1p}$$
            for all functions $x: \{1, ..., n\} \to K$ given by $x(i) := x_i$ for all $1 \leq i \leq n$. $\ell^p(K^n)$ is therefore an example of a finite-dimensional $L^p$-space.

            For each $p, q \in \N_{\geq 1} \cup \{+\infty\}$ such that $\frac1p + \frac1q = 1$, we have that:
                $$\ell^q(K^n) \cong \ell^p(K^n)^*_{\cont}$$
            by theorem \ref{theorem: L_p_space_duality}. By finite-dimensionality, each $\varphi \in \ell^p(K^n)^*_{\cont}$ is representable by some $y := (y_1, ..., y_n) \in \ell^q(K^n)$ in the sense that:
                $$\varphi(x) := x \cdot y := \sum_{i = 1}^n x_i y_i$$
            for all $x := (x_1, ..., x_n) \in \ell^p(K^n)$, where $\cdot$ is the usual dot product. The discrete version of H\"older's inequality thus reads:
                $$\abs{x \cdot y} \leq \norm{x}_{L^p(X, \mu)} \norm{y}_{L^q(X, \mu)}$$

            Each $\ell^p(K^n)$ is also complete with respect to the topology defined by its norm. Note also that when $p = q = 2$, $\ell^2(K^n)$ will be a Hilbert space.
        \end{example}

    \subsection{Sobolev spaces}