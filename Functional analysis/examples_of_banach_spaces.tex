\section{Examples of Banach and Hilbert spaces}
    \subsection{Continuous functions}
        Recall that if $X, Y$ are normed spaces then there will be an isometric isomorphism $\Hom_{K, \cont}(X^{\wedge}, Y) \xrightarrow[]{\cong} \Hom_{K, \cont}(X, Y)$ induced by the isometric inclusion $X \subseteq X^{\wedge}$ (lemma \ref{lemma: domain_completion}), and hence when $Y \cong K$, we shall have an isometric isomorphism:
            $$(Y^{\wedge})^*_{\cont} \xrightarrow[]{\cong} Y^*_{\cont}$$
            
        \begin{lemma}[Approximating continuous functions by compactly supported ones] \label{lemma: approximating_continuous_functions_by_compactly_supported_ones}
            Let $X$ be a locally compact and Hausdorff topological space. Then, with respect to the topology generated by $\norm{-}_{\infty}$, the inclusion:
                $$C^0_c(X)^{\wedge} \hookrightarrow C^0(X)$$
            is actually an isometric isomorphism. Consequently, we have an isometric isomorphism:
                $$C^0(X)^*_{\cont} \xrightarrow[]{\cong} (C^0_c(X)^{\wedge})^*_{\cont}$$
            induced by the inclusion above. 
        \end{lemma}
            \begin{proof}
                
            \end{proof}
            
        \begin{theorem}[Riesz-Markov-Kakutani representation theorem: measures are functionals] \label{theorem: measures_as_linear_functionals}
            
        \end{theorem}
            \begin{proof}
                
            \end{proof}

    \subsection{\texorpdfstring{$L^p$}{}-spaces} \label{subsection: L_p_spaces}
        \begin{convention}
            Let $(X, \mu)$ be a measure space. Unless specified to be otherwise, we shall be writing:
                $$L^p(X, \mu) := \left\{ f: X \to K \bigg\mid \norm{f}_{L^p(X, \mu)} := \left( \int_X \abs{f}^p d\mu \right)^{\frac1p} < +\infty \right\}$$
            for all $p \in \N$.

            For $p = \infty$, let us say that a function $f: X \to K$ is essentially bounded by some constant $M \in \R_{\geq 0}$ if and only if $\abs{f(x)} \leq M$ for $\mu$-almost every $x \in X$. For such functions $f$, we can define the following quantity:
                $$\norm{f}_{L^{\infty}(X, \mu)} := \inf\{ M \in \R_{\geq 0} \mid \abs{f(x)} \leq M \}$$
            We then define:
                $$L^{\infty}(X, \mu) := \left\{ f: X \to K \mid \norm{f}_{L^{\infty}(X, \mu)} < +\infty \right\}$$
        \end{convention}
    
        For what follows, recall that in a vector space $E$, a subset $K \subseteq E$ is said to be \textbf{convex} if and only if for every $x, y \in E$ and every $\lambda \in [0, 1]$, we have that:
            $$\lambda(x - y) + y \in K$$
        and a function $J: K \to \R$ is said to be \textbf{convex} if and only if:
            $$J( \lambda(x - y) + y ) \leq \lambda( J(x) - J(y) ) + J(y)$$
        for all $\lambda \in [0, 1]$. What the first definition is saying is that a subset $K$ of a vector space is convex if and only if given any $x, y \in K$, the line segment between those two points must also lie entirely within $K$; notice that points on said line segment are precisely given by:
            $$x_{\lambda} = \lambda(x - y) + y$$
        each for a specified $\lambda \in [0, 1]$, and as the parameter $\lambda$ varies, we travel from $x$ to $y$ (or vice versa). The second definition therefore is saying that a function is convex if and only if it preserves this property whereby line segments remain inside.

        It is also worth noting that convex sets are, by definition (since we are considering line segments in the definition), path-connected. Consequently, they are connected. 
        
        Also, if $(X, \mu)$ is any measure space (we are suppressing the $\sigma$-algebra) and $f: X \to \R$ is a measurable function, then let us write:
            $$\bbE[f] := \frac{1}{\mu(X)} \int_X f d\mu$$
        This is the \textbf{average value} of $f$ (or the \textbf{expected value} if the domain of $f$ is $[0, 1]$ and if $\mu(X) = 1$, because in that case $\mu$ would be a probability measure).
        \begin{lemma}[Convex functions are continuous]
            If $E$ is a finite-dimensional normed vector space and $K$ is a convex subset thereof, then any convex function $J: K \to \R$ will be continuous. 
        \end{lemma}
            \begin{proof}
                First of all, we claim that it is enough to show that any convex function $J: (a, b) \to \R$ is necessarily continuous. 
            \end{proof}
        \begin{lemma}[Jensen's inequality] \label{lemma: jensen_inequality}
            Let $(X, \mu)$ be a measure space and $J: \R \to \R$ be a convex function. Then, we shall have that:
                $$J( \bbE[f] ) \leq \bbE[ J \circ f ]$$
            for any real-valued $f \in L^1(X, \mu)$. 
        \end{lemma}
            \begin{proof}
                
            \end{proof}
        \begin{lemma}[H\"older's inequality]
            Let $1 \leq p \leq q \leq +\infty$ be such that:
                $$\frac1p + \frac1q = 1$$
            Let $(X, \mu)$ be a measure space and let $f \in L^p(X, \mu), g \in L^q(X, \mu)$. Then, we have the following inequality:
                $$\norm{f}_{L^p(X, \mu)} \norm{g}_{L^q(X, \mu)} \geq \norm{fg}_{L^1(X, \mu)}$$
            and hence:
                $$fg \in L^1(X, \mu)$$
        \end{lemma}
            \begin{proof}
                \todo[inline]{Apply Radon-Nikodym to find a new probability measure $\nu$ on $X$ (what do we use as the Radon-Nikodym derivative ?) for which we can apply Jensen's inequality to the convex function $\norm{-}^p$ (this is why $p \geq 0$ is crucial).}
            \end{proof}
        \begin{corollary}[$L^p$-spaces inclusions]
            If $(X, \mu)$ is a measure space (i.e. $\mu(X) < +\infty$) then:
                $$\forall 1 \leq p, q \leq +\infty: p \leq q \implies L^p(X, \mu) \subset L^q(X, \mu)$$
            if and only if $X$ is $\mu$-finite, i.e.:
                $$\mu(X) < +\infty$$ 
        \end{corollary}
            \begin{proof}
                
            \end{proof}

        \begin{lemma}[Minkowski's inequality] \label{lemma: minkowski_inequality}
            For any measure space $(X, \mu)$, the pairs:
                $$(L^p(X, \mu), \norm{-}_{L^p(X, \mu)})$$
            are normed spaces.
        \end{lemma}
            \begin{proof}
                \todo[inline]{This relies on H\"older's inequality.}
            \end{proof}
        \begin{theorem}[$L^p$-spaces are complete] \label{theorem: L_p_space_completeness}
            The normed spaces $(L^p(X, \mu), \norm{-}_{L^p(X, \mu)})$ are complete with respect to their metric topologies.
        \end{theorem}
            \begin{proof}
                \todo[inline]{Minkowski's inequality + some standard $\e$-$\delta$ convergence arguments.}
            \end{proof}

        \todo[inline]{$L^p$-spaces (infinite measures)}

        \begin{lemma}[Comparing $L^p$ and $L^q$-norms: finite-measure cases] \label{lemma: comparing_L_p_norms_for_finite_measures}
            Let $(X, \Sigma, \mu)$ be a measure space and choose $1 \leq p, q \leq +\infty$. If $p < q$ and $\mu(X)$ is finite, then:
                $$\norm{f}_{L^p(X, \mu)} \leq \mu(X)^{\frac1p - \frac1q} \norm{f}_{L^q(X, \mu)}$$
            Furthermore, the constant $\mu(X)^{\frac1p - \frac1q}$ is sharp.
        \end{lemma}
            \begin{proof}
                Let us apply H\"older's inequality to the pair of functions $1$ and $f^p$, using the pair of conjugate exponents $\frac{1}{1 - \left(\frac{q}{p}\right)^{-1}} = \frac{1}{1 - \frac{p}{q}} = \frac{q}{q - p}$ and $\frac{q}{p}$ (note that since $p < q$ by assumption, we indeed have that $\frac{q}{p} \geq 1$), which yields:
                    $$\norm{f}_{L^p(X, \mu)}^p = \int_X \abs{f}^p d\mu = \norm{ 1 \cdot f^p }_{L^1(X, \mu)} \leq \norm{1}_{L^{\frac{q}{q - p}}(X, \mu)} \norm{f^p}_{L^{\frac{q}{p}}(X, \mu)}$$
                Now, we have:
                    $$\norm{1}_{L^{\frac{q}{q - p}}(X, \mu)} = \left(\int_X \abs{1}^{\frac{q}{q - p}} d\mu\right)^{ \frac{1}{\frac{q}{q - p}} } = \mu(X)^{ \frac{q - p}{q} }$$
                    $$\norm{f^p}_{L^{\frac{q}{p}}(X, \mu)} = \left( \int_X \abs{f}^q d\mu \right)^{\frac{p}{q}} = \norm{f}_{L^q(X, \mu)}^p$$
                Lastly, since $p \geq 1$, the function $(-)^{\frac1p}: \R_{\geq 0} \to \R_{\geq 0}$ is monotonic. By putting everything together, we yield:
                    $$\norm{f}_{L^p(X, \mu)} \leq \mu(X)^{ \frac{q - p}{qp} } \norm{f}_{L^q(X, \mu)} = \mu(X)^{\frac1p - \frac1q} \norm{f}_{L^q(X, \mu)}$$
                which is as desired.
            \end{proof}

        \todo[inline]{$L^p$-spaces (infinite measures)}

        \begin{lemma}[$L^p$-space inclusions for infinite measures] \label{lemma: L_p_space_inclusions_infinite_measures}
            Let $(X, \Sigma, \mu)$ be a measure space and choose $1 \leq p, q \leq +\infty$. If $p < q$ and:
                $$\inf_{S \in \Sigma, S \not = \varnothing} \mu(S) > 0$$
            (i.e. there are no non-empty measure-zero sets), and:
                $$\mu(S) = +\infty$$
            then there will be a strict inclusion:
                $$L^p(X, \mu) \subset L^q(X, \mu)$$
        \end{lemma}
            \begin{proof}
                For each $n \in \N$, let:
                    $$X_{> n} := \{ x \in X \mid \forall f \in L^p(X, \mu): \abs{f(x)} > n \}$$
                Then, we have that:
                    $$\norm{f}_{L^p(X, \mu)}^p := \int_X \abs{f}^p d\mu \geq \sum_{n \geq 0} \int_{X_{> n}} \abs{ n \chi_{X_{> n}} }^p d\mu = \sum_{n \geq 0} \int_{X_{> n}} n^p d\mu = \sum_{n \geq 0} n^p \mu(X_{> n})$$
                From this, we gather that:
                    $$\forall n \in \N: \frac{1}{n^p} \norm{f}_{L^p(X, \mu)} \geq \mu(X_{> n})$$
                and hence:
                    $$n \gg 0 \implies 0 \geq \mu(X_{> n})$$
                and since $\mu$ takes values in $\R_{\geq 0}$, the above implies that $\mu(X_{> n}) = 0$ whenever $n \gg 0$. As we have assumed that:
                    $$\inf_{S \in \Sigma, S \not = \varnothing} \mu(S) > 0$$
                i.e.:
                    $$\mu(S) = 0 \iff S = \varnothing$$
                the fact that $\mu(X_{> n}) = 0$ whenever $n \gg 0$ now implies that:
                    $$n \gg 0 \implies X_{> n} = \varnothing$$
                This means that for $n \gg 0$, there are no functions $f \in L^p(X, \mu)$ for which there exists any $x\ in X$ such that $\abs{f(x)} > n$; in other words, any $f \in L^p(X, \mu)$ must be bounded, i.e. $\norm{f}_{\infty}$ is finite.

                Now, observe that since $1 \leq p < q$, the functions $(-)^p, (-)^q: \R_{\geq 0} \to \R_{\geq 0}$ are monotonic and we also have that:
                    $$(-)^q \leq (-)^p$$
                Next, let:
                    $$X_{\leq 1} := \{ x \in X \mid \forall f \in L^p(X, \mu): \abs{f(x)} \leq 1 \}$$
                and note that:
                    $$X = X_{\leq 1} \sqcup X_{> 1}$$
                which means that:
                    $$\int_X (-) d\mu = \int_{X_{\leq 1}} (-) d\mu + \int_{X_{> 1}} (-) d\mu$$
                We are now led to consider the following:
                    $$\norm{f}_{L^q(X, \mu)}^q := \int_X \abs{f}^q d\mu = \int_{X_{\leq 1}} \abs{f}^q d\mu + \int_{X_{> 0}} \abs{f}^q d\mu \leq \int_{X_{\leq 1}} \abs{f}^p d\mu + \norm{f}_{\infty} = \norm{f}_{L^p(X_{> 1}, \mu)}^p + \norm{f}_{\infty}$$
                Since $f \in L^p(X, \mu)$ and since $X_{> 1}$ is a measurable subset of $X$ (and also because $f$ is bounded \textit{a priori}), the expression $\norm{f}_{L^p(X_{> 1}, \mu)}^p + \norm{f}_{\infty}$ is finite, and hence we can conclude that $f \in L^q(X, \mu)$. We thus have:
                    $$L^p(X, \mu) \subset L^q(X, \mu)$$
                and the inclusion is strict because $\abs{f}^q = \abs{f}^p$ if and only if $\abs{f} = 1$ (supposing that $p < q$).
            \end{proof}

        \todo[inline]{$L^p$-duality}
        \begin{lemma}[The $L^1$-pairing] \label{lemma: L_1_pairing}
            Let $1 \leq p, q \leq +\infty$ be such that $\frac1p + \frac1q = 1$ and let $(X, \mu)$ be a measure space. There is an isometry:
                $$I_{p, q}: L^q(X, \mu) \xrightarrow[]{\cong} L^p(X, \mu)^*_{\cont}$$
            given by:
                $$I_{p, q}(g)[-] := \int_X \abs{(-) g} d\mu$$
            for all $g \in L^q(X, \mu)$.
        \end{lemma}
            \begin{proof}
                For all $g \in L^q(X, \mu)$, we have:
                    $$\norm{I_{p, q}(g)} := \sup_{f \in L^p(X, \mu) \setminus \{0\}} \frac{ I_{p, q}(g)[f] }{\norm{f}_{L^p(X, \mu)}} = \sup_{f \in L^p(X, \mu) \setminus \{0\}} \frac{ \int_X \abs{fg} d\mu }{\norm{f}_{L^p(X, \mu)}}$$
                By H\"older's Inequality, we know that $\int_X \abs{fg} d\mu \leq \norm{f}_{L^p(X, \mu)} \norm{g}_{L^q(X, \mu)}$. Using this, we get:
                    $$\norm{I_{p, q}(g)} = \sup_{f \in L^p(X, \mu) \setminus \{0\}} \frac{ \norm{f}_{L^p(X, \mu)} \norm{g}_{L^q(X, \mu)} }{\norm{f}_{L^p(X, \mu)}} = \norm{g}_{L^q(X, \mu)}$$ 
            \end{proof}
        \begin{theorem}[$L^p$-space duality] \label{theorem: L_p_space_duality}
            Let $1 \leq p, q \leq +\infty$ be such that $\frac1p + \frac1q = 1$ and let $(X, \mu)$ be a measure space.
            \begin{enumerate}
                \item For $1 < p, q < +\infty$, the isometry $I_{p, q}$ as in lemma \ref{lemma: L_1_pairing} will be an isometric isomorphism.
                \item When $p \not = q \in \{1, +\infty\}$, both $I_{1, \infty}$ and $I_{\infty, 1}$ will also be isometric isomorphisms if and only if $(X, \mu)$ is $\sigma$-finite, i.e. $\mu(X) < +\infty$.
            \end{enumerate}
        \end{theorem}
            \begin{proof}
                \begin{enumerate}
                    \item 
                    \item 
                \end{enumerate}
            \end{proof}
            
        \begin{example}[$\ell^p$-spaces] \label{example: ell_p_spaces}
            If $\Gamma$ is a discrete space and $\mu$ is the counting measure, then usually we shall write:
                $$\ell^p(\Gamma) := L^p(\Gamma, \mu)$$
            Note that now we have that:
                $$\int_\Gamma f d\mu := \sum_{\gamma \in \Gamma} f(\gamma)$$
            so $f: \Gamma \to K$ is $\mu$-integrable if and only if $\sum_{\gamma \in \Gamma} f < +\infty$, and hence elements of $\ell^p(\Gamma)$ are sequences $f := \{f_{\gamma}\}_{\gamma \in \Gamma}$ (which is nothing but a $\mu$-integrable function $f: \Gamma \to K$) such that $\norm{f}_p := \left( \sum_{\gamma \in \Gamma} \abs{f_{\gamma}}^p \right)^{\frac1p} < +\infty$. Equivalently, one can regard $\ell^p(\Gamma)$ as the vector subspace of $K^{\Gamma} := \prod_{\gamma \in \Gamma} K$ consisting of vectors $f := (f_{\gamma})_{\gamma \in \Gamma}$ such that $\norm{f}_p < +\infty$; note that we have:
                $$K^{\oplus \Gamma} \subseteq \ell^p(\Gamma) \subseteq K^{\Gamma}$$
            with the two inclusions being equalities if and only if $\Gamma$ is finite, since elements of $K^{\oplus \Gamma}$ are vectors with only finitely many non-zero entries, while if $\Gamma$ is infinite then the inclusions will be strict, because if $f := ( f_{\gamma} )_{\gamma \in \Gamma} \in K^{\Gamma}$ is such that $f_{\gamma} \not = 0$ for all $\gamma \in \Gamma$ then $\norm{f}_p \not < +\infty$.
        \end{example}

        Let us now consider an example where $L^p$-duality fails over a $\sigma$-infinite measure space, namely $\N$ with the counting measure.
        \begin{remark}
            For what follows, let us note that if $\Gamma$ is a countable discrete space then it will have a one-point compactification:
                $$\Gamma^{\wedge} := \Gamma \cup \{\infty\}$$
            which carries the profinite topology: every subset $U \subseteq \Gamma^{\wedge}$ containing $\infty$ is open, while the subspace $\Gamma \subset \Gamma^{\wedge}$ remains discrete, i.e. every subset of $\Gamma$ is automatically open. As the name suggests, $\Gamma^{\wedge}$ is a compact space, which is by virtue of being a profinite space; categorically, one has:
                $$\Gamma^{\wedge} \cong \projlim_{\text{$U \subseteq \Gamma$ finite}} U$$
            with the diagram that we are taking the limit over being cofiltered via inclusions.

            As continuous functions on compact spaces (e.g. profinite spaces) are automatically bounded, the space of sequences $\{a_{\gamma}\}_{\gamma \in \Gamma} \subset K$ that converge is then isomorphic to the subspace $C^0(\Gamma^{\wedge})$ of $\ell^{\infty}(\Gamma)$. Inside $C^0(\Gamma^{\wedge})$, we have the subspaces:
            \begin{itemize}
                \item $C^0(\Gamma)$ of sequences $\{a_{\gamma}\}_{\gamma \in \Gamma} \subset K$ that converge to $0$, and 
                \item $C^0_c(\Gamma)$ (sometimes denoted $c_{00}(\Gamma)$) of compactly supported sequences $\{a_{\gamma}\}_{\gamma \in \Gamma} \subset K$, i.e. sequences wherein $a_{\gamma} \not = 0$ only for finitely many $\gamma \in \Gamma$; $\Gamma = \N$, which is ordered, these are sequences such that $a_{\gamma} = 0$ for $\gamma \gg 0$. 
            \end{itemize}
        \end{remark}
            
        \begin{example}[Continuous duals of convergent sequence spaces] \label{example: continuous_duals_of_convergent_sequence_spaces}
            As a consequence of the Riesz-Markov-Kakutani Representation Theorem (theorem \ref{theorem: measures_as_linear_functionals}), which applies because $\N$ is compact and Hausdorff as a profinite space, we have an isometric isomorphism $\ell^1(\N) \xrightarrow[]{\cong} C^0(\N)^*_{\cont}$.
            
            Next, we claim that there is an isometric isomorphism $I_{1, \infty}: \ell^{\infty}(\N) \xrightarrow[]{\cong} \ell^1(\N)^*_{\cont}$, determined for all $g := \{g_n\}_{n \geq 0} \in \ell^{\infty}(\N)$ and all $f := \{f_n\}_{n \geq 0} \in \ell^1(\N)$, by $I_{1, \infty}(g)[f] := \int_{\N} \abs{fg} d\mu = \sum_{n \geq 0} \abs{ f_n g_n }$ (cf. theorem \ref{theorem: L_p_space_duality}). 
                
            We see now that $\ell^{\infty}(\N) \cong \ell^1(\N)^*_{\cont} \cong C^0(\N)^{**}_{\cont}$ is strictly larger than $C^0(\N)$, meaning in particular that $C^0(\N)$ is not reflexive. 
        \end{example}

        Now, even though norms on finite-dimensional vector spaces in general are all equivalent to one another \textit{a priori} (see proposition \ref{prop: norm_equivalence}), in the case of $\ell^p$-norms on finite-dimensional vector spaces, we can actually write down explicitly how much they differ from one another.
        \begin{lemma}[Sharp comparison of $\ell^p$-norms in finite dimensions] \label{lemma: sharp_comparison_of_ell_p_norms_in_finite_dimensions}
            Fix $n \in \N$ and let $1 \leq p < q < +\infty$. Then:
                $$c \norm{-}_p \leq \norm{-}_q \leq C \norm{-}_p$$
            as norms on $K^{\oplus n}$, wherein $c = n^{\frac1q - \frac1p}$ and $C = 1$. Furthermore, these constants are sharp.
        \end{lemma}
            \begin{proof}
                Let us write $\mu$ to mean the counting measure on the discrete space $\{1, ..., n\}$. Since $\mu(\{1, ..., n\}) = n$ is finite by construction, lemma \ref{lemma: comparing_L_p_norms_for_finite_measures} applies and tells us that if $p < q$ then $\norm{-}_p \leq n^{\frac1p - \frac1q} \norm{-}_q$, and that the constant $n^{\frac1p - \frac1q}$ is sharp, so we can take $c := n^{-(\frac1p - \frac1q)} = n^{\frac1p - \frac1p}$. Now, to find $C$, let us use to fact that norms homogeneous to assume without any loss of generality that $\norm{x}_q = 1$. This forces $\abs{x_i}^q \leq 1$ for all $1 \leq i \leq n$, and hence $p < q$ implies that $\norm{x}_q = \left( \int_{\{1, ..., n\}} \abs{x}^q d\mu \right)^{\frac1q} \leq \left( \int_{\{1, ..., n\}} \abs{x}^p d\mu \right)^{\frac1p} = \norm{x}_p$. Thus, we can take $C := 1$.
            \end{proof}

        \begin{example}[(In)separability of $\ell^p$ (finite $p$)] \label{example: (in)separability_of_ell_p_spaces_finite_p}
            Let $\Gamma$ be a discrete space.
            
            For $1 \leq p < +\infty$, the normed space $\ell^p(\Gamma)$ is separable if and only if $\Gamma$ is countable. In one direction, if $\Gamma$ is countable, then $\overline{\span \{e_{\gamma}\}_{\gamma \in \Gamma}} = \ell^p(\Gamma)$ - where $e_{\gamma}$ is the sequence with $1$ in the $\gamma^{th}$ entry and $0$ elsewhere, with respect to some ordering on $\Gamma$ - and hence the countable subset $\span \{e_{\gamma}\}_{\gamma \in \Gamma} \subset \ell^p(\Gamma)$ will be dense, thus making $\ell^p(\Gamma)$ separable by definition. Conversely, if $\ell^p$
        \end{example}
        \begin{example}[More about $\ell^2$-spaces]
            Let $\Gamma$ be a discrete space.
            
            Then, $\ell^2(\Gamma)$ will be separable if and only if $\Gamma$ is countable. We know from example \ref{example: (in)separability_of_ell_p_spaces_finite_p} that if $\Gamma$ is countable then $\ell^2(\Gamma)$ will be separable. To prove the converse claim, we use the fact that any orthonormal basis - hence any basis - of a separable Hilbert space is necessarily countable (see proposition \ref{prop: separable_hilbert_spaces_are_countably_dimensional}). 
        \end{example}

        \begin{example}[Inseparability of $\ell^{\infty}$]
            
        \end{example}

    \subsection{Harmonic analysis on locally compact Lie groups}

    \subsection{Existence of uniqueness of solutions to ODEs}

    \subsection{Sobolev spaces} \label{subsection: sobolev_spaces}