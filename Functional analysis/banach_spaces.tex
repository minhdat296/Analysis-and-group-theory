\section{Normed spaces}
    Throughout, $K$ shall be a complete archimedean field, so either $\R$ or $\bbC$, and the absolute value on $K$ will be denoted by $\abs{-}$. All vector spaces, unless stated otherwise, shall be $K$-vector spaces.

    \begin{convention}
        To conform to a traditional convention in the theory of integral equations, in calculus of variations, and in probability theory, the arguments of functions whose inputs are linear functionals will be placed between square brackets $[]$ as opposed to the usual parentheses $()$. For example, if $V, W$ are vector spaces and $T: V \to W$ is a linear map then the evaluation of the dual map $T^*: W^* \to V^*$ on functionals $\psi \in W^*$ will be written:
            $$T[\psi^*]$$
        instead of $T(\psi^*)$. This may also help with readability in certain cases. 
    \end{convention}
 
    \subsection{Generalities on topological vector spaces}
        \begin{definition}[Norms and normed spaces] \label{def: normed_spaces}
            \begin{enumerate}
                \item 
                \item 
                \item Moreover, there is a category $K\-\Vect_{\norm{-}}$ whose objects are normed $K$-vector spaces and whose objects are continuous $K$-linear maps between them. 
            \end{enumerate}
        \end{definition}
        \begin{lemma}[Norms are continuous] \label{lemma: norms_are_continuous}
            Let $\R_{\geq 0}$ be equipped with the subspace topology inherited from the norm topology on $\R$ and let $(X, \norm{-}_X)$ be a normed space. Then, the norm $\norm{-}_X$ will be a continuous function with respect to the norm topologies on $X$ and on $\R_{\geq 0}$.
        \end{lemma}
            \begin{proof}
                Choose a convergent sequence $\{x_n\}_{n \geq 0} \to x$ in $X$. This occurs if and only if:
                    $$\forall \e > 0: \exists N \in \N: n \geq N \implies \norm{x_n - x}_X < \e$$
                By the triangle inequality, we have that:
                    $$\norm{x_n - x}_X \geq \abs{ \norm{x_n}_X - \norm{x}_X }$$
                and hence:
                    $$\forall \e > 0: \exists N \in \N: n \geq N \implies \abs{ \norm{x_n}_X - \norm{x}_X } < \e$$
                This implies that the sequence $\{ \norm{x}_X \}_{n \geq 0}$ converges to $\norm{x}_X$ in $\R_{\geq 0}$, and hence $\norm{-}_X: X \to \R_{\geq 0}$ is a continuous function with respect to the norm topologies on $X$ and $\R_{\geq 0}$.
            \end{proof}

        \begin{proposition}[Equivalence of norms] \label{prop: norm_equivalence}
            Two norms $\norm{-}_1, \norm{-}_2$ on the same vector space $V$ are said to be \textbf{equivalent} if and only if there exist constants $c, C > 0$ such that:
                $$c \norm{-}_1 \leq \norm{-}_2 \leq C \norm{-}_1$$
            \begin{enumerate}
                \item Equivalence of norms on a given vector spaces $V$ is an equivalence relation on the set of norms on $V$.
                \item If two norms are equivalent then they define the same topology. The converse statement needs not hold.
                \item Furthermore, all norms on a finite-dimensional vector space are equivalent to one another.
            \end{enumerate}
        \end{proposition}
            \begin{proof}
                \begin{enumerate}
                    \item 
                    \item 
                    \item 
                \end{enumerate}
            \end{proof}
            
        \begin{proposition}[The operator norm] \label{prop: operator_norm}
            \begin{enumerate}
                \item Let $V, W$ be normed spaces and $\varphi: V \to W$ be a linear map. Then, we can define a norm on $\Hom_{K, \cont}(V, W)$, called the \textbf{operator norm}, given by:
                    $$\norm{\varphi}_{\Hom_{K, \cont}(V, W)} := \sup_{v \in V, v \not = 0} \frac{ \norm{\varphi(v)}_W }{ \norm{v}_V }$$
                \item Equivalently, the operator norm can be given by:
                    $$\norm{\varphi}_{\Hom_{K, \cont}(V, W)} = \sup_{v \in V, \norm{v}_V = 1} \norm{\varphi(v)}_W$$
                    $$\norm{\varphi}_{\Hom_{K, \cont}(V, W)} = \sup_{v \in V, 0 < \norm{v}_V \leq 1} \norm{\varphi(v)}_W$$
                    $$\norm{\varphi}_{\Hom_{K, \cont}(V, W)} = \inf\{ M \in \R_{\geq 0} \mid \norm{\varphi(v)}_W \leq M \norm{v}_V \}$$
            \end{enumerate}
        \end{proposition}
            \begin{proof}
                \begin{enumerate}
                    \item 
                    \item 
                \end{enumerate}
            \end{proof}
        \begin{lemma}[Bounded linear maps] \label{lemma: bounded_linear_maps}
            A linear map $\varphi: V \to W$ between normed spaces $V, W$ is continuous if and only if it is bounded, i.e. $\norm{\varphi}_{\Hom_{K, \cont}(V, W)} < +\infty$.
        \end{lemma}
            \begin{proof}
                Suppose firstly that $\varphi$ is bounded, say $\norm{\varphi}_{\Hom_{K, \cont}(V, W)} \leq M$ for some $M > 0$. To prove that $\varphi$ is continuous, we shall need to show that, for all $\e > 0$, if $\norm{v - v'}_V < \delta$ for some $\delta > 0$, then $\norm{\varphi(v - v')}_W < \e$. For this, consider the following, which holds by the definition of the operator norm:
                    $$\norm{\varphi(v - v')}_W \leq \norm{\varphi}_{\Hom_{K, \cont}(V, W)} \norm{v - v'}_V < M \delta$$
                Choosing $\delta := \frac{\e}{M}$ does the job.

                Conversely, suppose that $\varphi$ is continuous, say for all $\e > 0$, there exists $\delta > 0$ such that $\norm{v}_V < \delta$ implies $\norm{\varphi(v)}_W < \e$. At the same time, suppose for the sake of deriving a contradiction that $\varphi$ is unbounded. By proposition \ref{prop: operator_norm}, this means that:
                    $$\e > \norm{\varphi(v)}_W \geq M \norm{v}_V$$
                for all $M \geq 0$, all $\e > 0$, and all $v \in V$ such that $\norm{v}_V < \delta$. This is clearly nonsense, so we have a contradiction. Thus, $\varphi$ must be bounded.
            \end{proof}

        \begin{definition}[Short linear maps] \label{def: short_linear_maps}
            We say that a (continuous) linear map $\varphi: V \to W$ between two normed spaces $V, W$ is \textbf{short} (or is a \textbf{linear contraction}) if and only if $\norm{\varphi}_{\Hom_{K, \cont}(V, W)} \leq 1$. 
        \end{definition}
        \begin{definition}[Isometries] \label{def: isometries}
            Let $(X, d_X), (Y, d_Y)$ be metric spaces. A continuous map $f: X \to Y$ is said to be \textbf{isometric} (or to be an \textbf{isometry}) if $\norm{-}_X = \norm{-}_Y \circ f$. 
        \end{definition}
        \begin{proposition}[$K\-\Ban_{\leq 1}$] \label{prop: category_of_banach_spaces_and_short_linear_maps}
            There is another category of Banach spaces, denoted by $K\-\Ban_{\leq 1}$, whose objects are Banach spaces and whose morphisms are short linear maps. Isomorphisms in this category are linear isometric homeomorphisms (equivalently, surjective short isomorphisms); let us refer to these as \textbf{short isomorphisms}.
        \end{proposition}
            \begin{proof}
                
            \end{proof}
        Clearly, $K\-\Ban_{\leq 1}$ is a non-full subcategory of $K\-\Ban$ with the same class of objects. It will also turn out to be the case that $K\-\Ban_{\leq 1}$ is a much nicer category than the larger category $K\-\Ban$ from an algebraic viewpoint, particularly because it is both complete and cocomplete. However, in analysis, one can not be content with just linear maps of norms $\leq 1$, so one is forced to study the more \say{defective} category $K\-\Ban$.
            
        \begin{example}[An unbounded linear map] \label{example: d/dx_is_unbounded}
            Let $X$ be a compact metric space and let $D: C^1(X) \to C^0(K)$ be the linear map given by:
                $$D(f)|_x := \frac{d}{dx} f(x)$$
            We claim that this is an unbounded operator with respect to the sup-norms on $C^1(X)$ and $C^0(X)$.

            To see why, pick $f, g \in C^1(X)$ such that $\norm{f - g}_{\infty} < \delta$ for some arbitrary $\delta > 0$, and then consider the following, which holds for all $\e > 0$:
                $$
                    \begin{aligned}
                        \norm{Df|_x - Dg|_x}_{\infty} & = \norm{ \frac{d}{dx} f(x) - \frac{d}{dx} g(x) }_{\infty}
                        \\
                        & = \lim_{h \to 0} \frac1h \norm{ ( f(x + h) - f(x) ) - ( g(x + h) - g(x) ) }_{\infty}
                        \\
                        & = \lim_{h \to 0} \frac2h ( \norm{f}_{\infty} + \norm{g}_{\infty} )
                        \\
                        & \geq \e
                    \end{aligned}
                $$
            wherein to get the second equality, we made use of the fact that norms are continuous (lemma \ref{lemma: norms_are_continuous}), and to get the last inequality, we made use of the fact that continuous functions on compact spaces are bounded, which implies via the definition of the sup-norm that that $\norm{f}_{\infty}, \norm{g}_{\infty}$ are both finite. We therefore can not pick any $\delta > 0$ such that $\norm{f - g}_{\infty} < \delta$ would imply $\norm{Df|_x - Dg|_x}_{\infty} < \e$ for all $\e > 0$, so $D$ is discontinuous. By lemma \ref{lemma: bounded_linear_maps}, it is therefore unbounded.

            Another way to see that $D$ is unbounded is to notice that $D(e^{n x}) = n e^{n x}$, and since $X$ is compact, we can set $M := \norm{e^{n x}}_{\infty} < +\infty$ is finite as a consequence of $e^{n x}$ being continuous. This then implies that $\norm{D}_{\Hom_{K, \cont}(C^1(X), C^0(X))} \geq n \norm{e^{n x}}_{\infty}$ for all $n \geq 0$. Hence, $D$ is unbounded.
        \end{example}
    
        \begin{lemma}[Colimits of normed spaces] \label{lemma: colimits_of_normed_spaces}
            The category $K\-\Vect_{\norm{-}}$ has the following colimits.
            \begin{enumerate}
                \item \textbf{(Epimorphisms):} Let $\pi: X \to Q$ be a continuous quotient map of normed spaces. Elements of $\bar{x} \in Q$ (for each of which there is an element $x \in X$, due to $\pi$ being surjective) can be written as cosets:
                    $$\bar{x} := x + \ker \pi$$
                for any choice of representative $x \in \pi^{-1}(\bar{x})$. From this and from the fact that $Q$ has the finest possible topology such that $\pi$ is continuous, one sees that:
                    $$\norm{\bar{x}}_Q := \inf_{y \in \ker \pi, y \not = 0} \norm{x - y}_E = \dist(x, \ker \pi)$$
                \item \textbf{(Direct sums):} If $V, W$ are normed spaces, then $V \oplus W$ can also be endowed with a norm, namely the following:
                    $$\norm{(v, w)}_{V \oplus W} := \norm{v}_V + \norm{w}_W$$
            \end{enumerate}
        \end{lemma}
            \begin{proof}
                \begin{enumerate}
                    \item 
                    \item 
                \end{enumerate}
            \end{proof}

        \todo[inline]{$p$-direct sums}

        \begin{convention}[$\ell^p$ direct sums] \label{conv: ell_p_direct_sums}
            If $V, W$ are normed spaces and $1 \leq p \leq +\infty$, then it is typical to write:
                $$V \oplus_p W$$
            to mean $V \oplus W$ (the usual algebraic direct sum, \textit{sans topologie}) equipped with the norm given for all $(v, w) \in V \oplus W$ by:
                $$
                    \norm{(v, w)}_{V \oplus_p W} :=
                    \begin{cases}
                        \text{$(\norm{v}_V^p + \norm{w}_W^p)^{\frac1p}$ if $1 \leq p < +\infty$}
                        \\
                        \text{$\max\{ \norm{v}_V, \norm{w}_W \}$ if $p = +\infty$}
                    \end{cases}
                $$
            (cf. \cite[Example, p. 24]{litvak_functional_analysis_notes}). Let us refer to $V \oplus_p W$ as the $p$-direct sum of $V$ and $W$.
        \end{convention}

    \subsection{Banach spaces}
        \begin{definition}[Banach spaces] \label{def: banach_spaces}
            A \textbf{Banach space} (over $K$) is a complete normed $K$-vector space. There is also a category $K\-\Ban$ of Banach spaces, which is the full subcategory of $K\-\Vect_{\norm{-}}$, wherein objects are Banach spaces.
        \end{definition}
        \begin{lemma}[Closed subspaces of Banach spaces] \label{lemma: closed_subspaces_of_banach_spaces}
            If $X$ is a Banach space and $Z \subseteq X$ is a closed vector subspace, then $Z$ will also be a Banach space.
        \end{lemma}
            \begin{proof}
                Closed subspaces of any complete metric spaces are complete too, as they contain all limit points by definition.
            \end{proof}
        
        \todo[inline]{Basic examples of Banach spaces.}
        
        \begin{example}[Finite-dimensional normed spaces] \label{example: finite_dimensional_normed_spaces_are_banach}
            
        \end{example}
        \begin{example}[Bounded functions] \label{example: bounded_functions}
            
        \end{example}
        \begin{example}[$\ell^{\infty}(\N)$: bounded countable sequences] \label{example: bounded_countable_sequences}
            Let $\ell^{\infty}(\N)$ consist of bounded countable sequences $\{x_n\}_{n \geq 0} \subset K$. These are equivalently bounded functions $x: \N \to K$ given by $x(n) := x_n$, and if we equip $\N$ with the discrete topology, $\ell^{\infty}(\N)$ will be nothing but $C^b(\N)$. Thus, $\ell^{\infty}(\N)$ is a Banach space.
        \end{example}
        \begin{example}[$\ell^0(\N)$: convergent countable sequences] \label{example: convergent_sequences}
            $\ell^0(\N)$, the vector subspace of $\ell^{\infty}(\N)$ spanned by \textit{convergent} (hence bounded) countable sequences in $K$ is a closed subspace, and hence also a Banach space.
        
            To prove this, observe firstly that $\ell^0(\N) = C^0(\N \cup \{+\infty\})$, where $\N \cup \{+\infty\}$ is the one-point compactification of $\N$. Continuous functions on compact spaces (such as $\N \cup \{+\infty\}$) are bounded, and linear combinations of continuous functions are still continuous functions, so $\ell^0(\N)$ is a vector subspace, and hence a normed subspace of $\ell^{\infty}(\N)$. To show that this subspace is closed, let $\{f_n\}_{n \geq 0} \subset \ell^0(\N)$ be a sequence that converges to some $f \in \ell^{\infty}(\N)$ with respect to $\norm{-}_{\infty}$. We shall need to show that $f$ is continuous, so let us pick an arbitrary convergent sequence $\{x_m\}_{m \geq 0} \to x$ in $\N \cup \{+\infty\}$ and then consider the following. Because each $f_n: \N \cup \{+\infty\} \to \bbK$ is continuous, we have that $\{f_n(x_m)\}_{n, m \geq 0} \to \{f_n(x)\}_{n \geq 0}$ with respect to $\abs{-}_{\bbK}$. At the same time, because $\{f_n\}_{n \geq 0} \to f$, we have that $\{f_n(x_m)\}_{n \geq 0} \to f(x_m)$ for each $m$, and also that $\{f_n(x)\}_{n \geq 0} \to f(x)$, both with respect to $\abs{-}_{\bbK}$. For every $\e > 0$, there then exist $N \in \N$ such that $m, n \geq N$ shall imply that:
                $$
                    \begin{aligned}
                        \abs{f(x_m) - f(x)}_{\bbK} & \leq \abs{ f(x_m) - f_n(x_m) }_{\bbK} + \abs{f_n(x_m) - f(x)}_{\bbK}
                        \\
                        & \leq \abs{ f(x_m) - f_n(x_m) }_{\bbK} + \abs{f_n(x_m) - f_n(x)}_{\bbK} + \abs{f_n(x) - f(x)}_{\bbK}
                        \\
                        & < \frac13 \e + \frac13 \e + \frac13 \e
                        \\
                        & = \e
                    \end{aligned}
                $$
            This shows that $f$ is continuous, so we are done.
        \end{example}
        \begin{example}[$c_0(\N)$: null sequences] \label{example: sequences_converging_to_zero}
            
        \end{example}
        \begin{example}[$c_{00}(\N)$: eventually zero sequences] \label{example: eventually_zero_sequences}
            
        \end{example}

        Another prominent class of examples consists of so-called \say{$L^p$-spaces}. We will encounter such entities in subsection \ref{subsection: L_p_spaces}.

        \begin{lemma}[Some colimits of Banach spaces] \label{lemma: colimits_of_banach_spaces}
            The full subcategory $K\-\Ban \subset K\-\Vect$ is closed under epimorphisms and finite direct sums. 
        \end{lemma}
            \begin{proof}
                This is nothing but a corollary of lemma \ref{lemma: colimits_of_normed_spaces}.
            \end{proof}
    
        Let us now discuss some fundamental theorems in the theory of Banach spaces, namely the Uniform Boundedness Principle, the Open Mapping Theorem, and the Closed Graph Theorem, as well as some of their implications regarding the homological structure of the category $K\-\Ban$, particularly how exactly\footnote{Pun intended.} this category deviates from being an abelian category. These are theorems \ref{theorem: uniform_boundedness}, \ref{theorem: open_mapping}, and \ref{theorem: closed_graph} respectively.
        \begin{lemma}[Suprema of averages] \label{lemma: suprema_of_averages}
            Let $X, Y$ be normed spaces and let $T: X \to Y$ be a continuous linear map between them. Fix $\e > 0$ and a centre $x \in X$. Then:
                $$\sup_{x' \in \B_{\e}(x)} \norm{T(x)}_Y \geq \e \norm{T}_{\Hom_{K, \cont}(X, Y)}$$
        \end{lemma}
            \begin{proof}
                Since every point $x' \in \B_{\e}(x)$ is of distance $< \e$ away from $x$, taking a supremum over all such points is the same as taking a supremum over all translations $x \pm \xi$, with $\xi \in \B_{\e}(0)$; essentially, we are taking $\xi := x' - x$. Thus, consider the following:
                    $$
                        \begin{aligned}
                            & \sup_{x' \in \B_{\e}(x)} \norm{T(x)}_Y
                            \\
                            = & \sup_{\xi \in \B_{\e}(0)} \max\left\{ \norm{T(x \pm \xi)}_Y \right\}
                            \\
                            \geq & \sup_{\xi \in \B_{\e}(0)} \frac12\left( \norm{T(x + \xi)}_Y + \norm{T(x - \xi)}_Y \right) \: \text{(maxima are larger than or equal to averages)}
                            \\
                            \geq & \sup_{\xi \in \B_{\e}(0)} \norm{T(\xi)}_Y \: \text{(triangle inequality)}
                            \\
                            = & \sup_{\xi \in \B_{\e}(0)} \norm{\xi}_X \norm{T}_{\Hom_{K, \cont}(X, Y)}
                            \\
                            = & \e \norm{T}_{\Hom_{K, \cont}(X, Y)}
                        \end{aligned}
                    $$
            \end{proof}
        \begin{theorem}[Uniform Boundedness Principle] \label{theorem: uniform_boundedness}
            Let $X$ be a Banach space and $Y$ be a normed space, and let $\{T_n\}_{n \geq 0}$ be a sequence in $\Hom_{K, \cont}(X, Y)$.
            \begin{enumerate}
                \item \textbf{(\textit{Na\"ive} uniform boundedness):} If the sequence is \textbf{pointwise bounded} in the sense that:
                    $$\forall x \in X: \sup_{n \geq 0} \norm{ T_n(x) }_Y < +\infty$$
                then the sequence will automatically be uniformly bounded, in the sense that:
                    $$\sup_{n \geq 0} \norm{T_n}_{\Hom_{K, \cont}(X, Y)} < +\infty$$
                \item \textbf{(Strong uniform boundedness\footnote{Also called the Banach-Steinhaus Theorem.}):} In fact, the same assertion remains true should the sequence $\{T_n\}_{n \geq 0}$ be pointwise bounded only on a non-meagre subset of $X$. 
            \end{enumerate}
        \end{theorem}
            \begin{proof}
                \begin{enumerate}
                    \item Suppose for the sake of deriving a contradiction that $\{T_n\}_{n \geq 0}$ is \textit{not} uniformly bounded. Without any loss of generality, suppose also that $\norm{T_n}_{\Hom_{K, \cont}(X, Y)} \geq \frac{1}{\delta^n}$ for every $n \geq 0$ and some $\delta > 1$. We will now attempt to show that our initial assumption will lead to a contradiction by showing that $\sup_{n \geq 0} \norm{T_n}_{\Hom_{K, \cont}(X, Y)}$ must now be bounded above by a finite number. By lemma \ref{lemma: suprema_of_averages}, we know that:
                        $$\sup_{n \geq 0} \sup_{\xi \in \B_{\e}(0)} \norm{T_n(\xi)}_Y \geq \sup_{n \geq 0} \e \norm{T_n}_{\Hom_{K, \cont}(X, Y)} \geq \sup_{n \geq 0} \e \frac{1}{\delta^n} = \e$$
                    for all $\e > 0$. Now, observe that because $\{T_n\}_{n \geq 0}$ is pointwise bounded, $\sup_{n \geq 0} \norm{T_n(\xi)}_Y$ is finite, and because each $T_n$ is continuous and hence bounded, each $\sup_{\xi \in \B_{\e}(0)} \norm{T_n(\xi)}_Y$ is also finite. Thus:
                        $$\sup_{n \geq 0} \sup_{\xi \in \B_{\e}(0)} \norm{T_n(\xi)}_Y < +\infty$$
                    but at the same time, this contradicts what we have above, which is that:
                        $$\forall \e > 0: \sup_{n \geq 0} \sup_{\xi \in \B_{\e}(0)} \norm{T_n(\xi)}_Y \geq \e$$
                    Therefore, $\{T_n\}_{n \geq 0}$ is uniformly bounded.
                    \item This version relies on theorem \ref{theorem: baire_category}. 
                \end{enumerate}
            \end{proof}
        \begin{remark}
            Let $X$ be a Banach space and $Y$ be a normed space, and let $\{T_n\}_{n \geq 0}$ be a sequence in $\Hom_{K, \cont}(X, Y)$. Note that if $\sup_{n \geq 0} \norm{ T_n(x) }_Y \leq M$, say, for all $x \in X$, then because:
                $$\norm{T_n}_{\Hom_{K, \cont}(X, Y)} := \sup_{x \in X \setminus \{0\}} \frac{ \norm{T_n(x)}_Y }{ \norm{x}_X }$$
            we are only guaranteed that:
                $$\sup_{n \geq 0} \norm{T_n}_{\Hom_{K, \cont}(X, Y)} \leq M \norm{x}_X$$
        \end{remark}
        The following is an important corollary to the Uniform Boundedness Principle.
        \begin{corollary}[Compact limits of pointwise convergent sequences of continuous linear maps] \label{coro: compact_limits_of_pointwise_convergent_sequences_of_continuous_linear_maps}
            Let $X$ be a Banach space and $Y$ be a normed space. Define $T \in \Hom_K(X, Y)$ by:
                $$\forall x \in X: \lim_{n \to +\infty} \norm{T_n(x) - T(x)}_Y = 0$$
            Then $T$ will be bounded, and hence $T \in \Hom_{K, \cont}(X, Y)$. Furthermore, $\{T_n|_Z\}_{n \geq 0} \to T|_Z$ with respect to $\norm{-}_{\Hom_{K, \cont}(X, Y)}$ on any \textit{compact} subspace $Z \subseteq X$.
        \end{corollary}
            \begin{proof}
                In a metric space, a convergent sequence is automatically bounded, so the convergent sequence $\{T_n(x)\}_{n \geq 0} \subset Y$ must be bounded. By theorem \ref{theorem: uniform_boundedness}, this implies that $\{T_n\}_{\geq 0} \subset \Hom_{K, \cont}(X, Y)$ is also a bounded sequence. Set $M := \sup_{n \geq 0} \norm{T_n}_{\Hom_{K, \cont}(X, Y)}$. This implies that:
                    $$\norm{T_n(x)}_Y \leq M \norm{x}_X$$
                for every $n \geq 0$ and $x \in X$. Then, consider the following for $n$ sufficiently large so that $\norm{T(x) - T_n(x)}_Y < \e$ for all $\e > 0$:
                    $$\norm{T(x)}_Y \leq \norm{T(x) - T_n(x)}_Y + \norm{T_n(x)}_Y < \e + M \norm{x}_X$$
                As $\e > 0$ is arbitrary, this implies that $\norm{T(x)}_Y \leq M \norm{x}_X$ for all $x \in X$, and hence $\norm{T}_{\Hom_{K, \cont}(X, Y)} \leq M$.

                Now, compact subsets of metric spaces are closed and bounded by the Heine-Borel Theorem. If $Z \subseteq X$ is a compact subset wherein $\norm{z}_X \leq C$ for all $z \in Z$. Then, we shall have:
                    $$\norm{T_n|_Z - T|_Z}_{\Hom_{K, \cont}(X, Y)} \leq C \sup_{z \in Z, \norm{z} \leq 1} \norm{ T_n(z) - T(z) }_Y < \e$$
                for $n \geq 0$ sufficiently large and for all $\e > 0$. This shows that $\{T_n|_Z\}_{n \geq 0}$ converges uniformly to $T|_Z$ for $Z \subseteq X$ compact.
            \end{proof}
        \begin{definition}[Compact limits] \label{def: compact_limits}
            We shall refer to $T$ as in corollary \ref{coro: compact_limits_of_pointwise_convergent_sequences_of_continuous_linear_maps} as the \textbf{compact limit} of the pointwise convergent sequence $\{T_n\}_{n \geq 0}$.
        \end{definition}
        Later on, this corollary will be used for proving that $K\-\Ban$ is enriched over itself (see lemma \ref{lemma: hom_between_banach_spaces_are_banach_spaces}). Corollary \ref{coro: compact_limits_of_pointwise_convergent_sequences_of_continuous_linear_maps} prompts the notion of compact linear maps. See proposition \ref{prop: compact_linear_maps}.

        \begin{theorem}[Open Mapping Theorem] \label{theorem: open_mapping}
            Surjective continuous linear maps between Banach spaces are open.
        \end{theorem}
            \begin{proof}
                \todo[inline]{Prove Open Mapping Theorem}
            \end{proof}
        \begin{corollary}[Linear homeomorphisms] \label{coro: linear_homeomorphisms}
            Any bijective continuous linear map between Banach spaces is automatically a (linear) homeomorphism.
        \end{corollary}
            \begin{proof}
                Let $\varphi: X \to Y$ be a bijective continuous linear map. To prove that $\varphi^{-1}$ is continuous, we must show that $\varphi(U) = (\varphi^{-1})^{-1}(U) \subseteq Y$ is open for every open subspace $U \subseteq X$ By theorem \ref{theorem: open_mapping}, $\varphi$ is open, so we are already done.
            \end{proof}
            
        \begin{lemma}[Extensions of Banach spaces] \label{lemma: banach_spaces_extensions}
            \begin{enumerate}
                \item 
                \item 
            \end{enumerate}
        \end{lemma}
            \begin{proof}
                \begin{enumerate}
                    \item 
                    \item 
                \end{enumerate}
            \end{proof}
        For a specific examples of monomorphisms of Banach spaces, both split and non-split, see subsection \ref{subsection: L_p_spaces}. Let us also remark that recognising split monomorphisms of Banach spaces is a deceptively difficult problem in practice. For instance, it is still unknown whether $L^1([0, 1])$ has any complemented Banach subspace. 

        \begin{example}[Counter-example: topologically nilpotent operators] \label{example: topologically_nilpotent_operators}
            Let $X$ be a Banach space and let $T \in \End_{K, \cont}(X)$. We say that $T$ is \textbf{topologically nilpotent} (also said to be \say{quasi-nilpotent}) if and only if:
                $$\lim_{n \to +\infty} \norm{T^n}_{\End_{K, \cont}(X)}^{\frac1n} = 0$$
            We claim that such a quasi-nilpotent operator $T$ is surjective if and only if $X \cong 0$.

            Of course, if $X \cong 0$ then it will be discrete, so the only continuous linear endomorphism will be the only linear endomorphism thereon, namely the zero map. The zero map $0: 0 \to 0$ is trivially surjective and quasi-nilpotent, so we are done.

            Conversely, suppose that $T \in \End_{K, \cont}(X)$ is quasi-nilpotent and surjective. By definition, this means that for all $\e > 0$ and all sufficiently large $n \geq 0$, we have that $\e > \norm{T^n}_{\End_{K, \cont}(X)}^{\frac1n}$. At the same time, because $T$ is surjective, we see via theorem \ref{theorem: open_mapping} that for each $y \in X$, there exists $x \in X$ and $C > 0$ such that $\norm{x}_X \leq C \norm{y}_X$ and $T(x) = y$, from which we infer that:
                $$\norm{x}_X \leq C \norm{T(x)}_X \leq C \norm{T}_{\End_{K, \cont}(X)} \norm{x}_X$$
            and hence:
                $$\norm{T}_{\End_{K, \cont}(X)} \geq \frac1C$$
            
        \end{example}

        \begin{theorem}[Closed Graph Theorem: continuous linear maps reflect limits] \label{theorem: closed_graph}
            \todo[inline]{Closed Graph Theorem}
        \end{theorem}
            \begin{proof}
                
            \end{proof}

    \subsection{The Hahn-Banach Theorem and duality}
        One fundamental problem in the theory of normed spaces is the question of whether the dual space may actually just be trivial. We know this to not be true always (e.g. for any measure space $(X, \mu)$, the spaces $L^p(X, \mu)$ and $L^q(X, \mu)$ are both infinite-dimensional - hence non-trivial - and are dual if $\frac1p + \frac1q = 1$) and in fact, as the Hahn-Banach Theorem will show, the dual space tends to be quite large, provided that the normed is \say{controlled} somehow.

        For convenience, let us begin by introducing the following terminology:
        \begin{definition}[Sublinearity] \label{def: sublinearity}
            A function $\rho: E \to \R$ on a vector space $E$ is said to be \textbf{sublinear} if and only if:
            \begin{itemize}
                \item $\rho$ satisfies the triangle inequality, i.e.:
                    $$\forall x, y \in X: \rho(x + y) \leq \rho(x) + \rho(y)$$
                \item $\rho$ preserves scalar multiplication, i.e.:
                    $$\forall \lambda \in K: \forall x \in E: \rho(\lambda x) = \lambda \rho(x)$$
            \end{itemize}
        \end{definition}
        \begin{example}
            Norms and semi-norms are sublinear. 
        \end{example}
        \begin{definition}[Dominance] \label{def: dominance}
            A real-valued function $\rho: X \to \R$ on a set $X$ is said to \textbf{dominant} another such function $f: X \to \R$ if:
                $$\forall x \in X: f(x) \leq \rho(x)$$
            In such a situation, we shall write:
                $$f \leq \rho$$
        \end{definition}

        The following lemma is very important.
        \begin{lemma}[Sublinear functions extend finitely] \label{lemma: sublinear_functions_extend_finitely}
            Let $K := \R$ and let $E$ be an $\R$-vector space. Let $\rho: E \to \R$ be a sublinear function that dominates an $\R$-linear functional:
                $$\varphi_0: E_0 \to \R$$
            defined on some subspace $E_0 \subseteq E$. Then, there shall exist an extension:
                $$\varphi_1: E_1 \to \R$$
            (i.e. $\varphi_1|_{E_0} = \varphi_0$) that remains linear and dominated by $\rho$, where $E_1 \subseteq E$ is any subspace such that $\dim E_1/E_0 < +\infty$. 
        \end{lemma}
            \begin{proof}
                We can prove this lemma by proving that for any $x_1 \in E \setminus E_0$, there shall exist an extension $\varphi_1: E_0 \oplus \R x_1 \to \R$ that remains linear and dominated by $\rho$. One can then repeat the process $\dim E_1/E_0$ times to get the full assertion. To this end, we shall need to specify how the expressions of the kind below are given:
                    $$\varphi_1( x + \lambda x_1 )$$
                for all $x \in E$ and $\lambda \in \R$. Consider, then, the following, which holds due to definition \ref{def: sublinearity}:
                    $$\rho(x + y) = \rho( x + \lambda x_1 - \lambda x_1 + y ) \leq \rho(x + \lambda x_1) + \rho(-\lambda x_1 + y)$$
                for all $x, y \in E$ and all $\lambda \in \R$. That $\varphi_0$ is linear and that $\varphi_0 \leq \rho$ together tell us that:
                    $$\varphi_0(x) + \varphi_0(y) = \varphi_0(x + y) \leq \rho(x + y)$$
                and hence:
                    $$\varphi_0(x) + \varphi_0(y) \leq \rho(x + \lambda x_1) + \rho(-\lambda x_1 + y)$$
                for all $x, y \in E_0$. Notice, then, that if we were to let:
                    $$\varphi_1(x + \lambda x_1) := \varphi_0(x) + \lambda \alpha_1$$
                for some choice of $\alpha_1 \in \R$ then because we would like:
                    $$\varphi_1(x + \lambda x_1) \leq \rho(x + \lambda x_1)$$
                we will have to choose $\alpha_1$ to be such that:
                    $$\varphi_0(x) + \lambda \alpha_1 \leq \rho(x + \lambda x_1) \iff \lambda \alpha_1 \leq \rho(x + \lambda x_1) - \varphi_0(x)$$
                for all $x \in E$. Since we only need to specify $\alpha_1 := \varphi_1(x_1)$, we can simply set $\lambda := 1$ and then choose:
                    $$\alpha_1 := \inf_{x \in E} \left( \rho(x + x_1) - \varphi_0(x) \right)$$
            \end{proof}
        \begin{remark}[Some comments about the proof]
            One key detail about lemma \ref{lemma: sublinear_functions_extend_finitely} (and hence also theorem \ref{theorem: hahn_banach}, which depends on said lemma) is that it relies crucially on everything being defined over a totally ordered archimedean field - so that theidea that a sublinear function can dominate a functional would even make sense - and on said archimedean field being complete, so that we can consider infima. 
        \end{remark}
        The Hahn-Banach Theorem tells us that $\R$-linear functionals actually extend all the way by a certain limiting procedure. 
        \begin{theorem}[Hahn-Banach: non-triviality of dual spaces] \label{theorem: hahn_banach}
            Let $K := \R$ and let $E$ be an $\R$-vector space. Let $\rho: E \to \R$ be a sublinear function that dominates an $\R$-linear functional:
                $$\varphi_0: E_0 \to \R$$
            defined on some subspace $E_0 \subseteq E$. Then, there shall exist an extension:
                $$\varphi_{\infty}: E \to \R$$
            (i.e. $\varphi_{\infty}|_{E_0} = \varphi_0$) that remains linear and dominated by $\rho$.

            Since $\varphi_{\infty} \in E^*$, this means that the dual space $E^*$ is non-trivial, since it is known that by picking $E_0$ to be finite-dimensional, one can always be guaranteed that there is some non-trivial $\varphi_0 \in E_0^*$ (and hence the extension $\varphi_{\infty}$ is also non-trivial).
        \end{theorem}
            \begin{proof}
                By lemma \ref{lemma: sublinear_functions_extend_finitely}, we know that there is an ascending chain of subspaces of $E$:
                    $$E_0 \subset E_1 \subset E_2 \subset ...$$
                which is possibly non-terminating and is such that $\dim E_{i + 1}/E_i = 1$ for all $i \in \N$, and each term $E_i$ comes equipped with a linear extension $\varphi_i: E_i \to \R$ that is dominated by the given sublinear function $\rho$. It is clear that by construction, we also have that:
                    $$\varphi_{i + 1|_{E_i}} = \varphi_i$$
                and so we have a filtered diagram $\{(E_i, \varphi_i)\}_{i \in \N}$ in the slice category $\R\-\Vect_{/\R}$. The category $\R\-\Vect$ is cocomplete, and slices of cocomplete categories are themselves cocomplete, so the (filtered) colimit:
                    $$(E_{\infty}, \varphi_{\infty}) := \indlim_{i \in \N} (E_i, \varphi_i)$$
                exists in $\R\-\Vect_{/\R}$. It remains to show that $E_{\infty} \cong E$ and $\varphi_{\infty} \leq \rho$.

                To prove that $E_{\infty} \cong E$, let us suppose to the contrary (for the sake of deriving a contradiction) that one can identify $E_{\infty}$ with a \textit{proper} subspace of $E$. This would imply that there exists some $x \in E \setminus E_{\infty}$. Should we also have that $\varphi_{\infty} \leq \rho$, then we know by lemma \ref{lemma: sublinear_functions_extend_finitely} that one can then simply extend the linear functional $\varphi_{\infty}$ once more, to a linear functional $\varphi_{\infty + 1}: E_{\infty} \oplus \R x \to \R$ such that $\varphi_{\infty + 1} \leq \rho$. The universal property of colimits guarantees, however, that there would be a unique isomorphism $(E_{\infty}, \varphi_{\infty}) \xrightarrow[]{\cong} (E_{\infty} \oplus \R x, \varphi_{})$ in $\R\-\Vect_{/\R}$, which is clearly nonsensical as $x \not = 0$, so we indeed have that $E_{\infty} \cong E$.

                To prove that $\varphi_{\infty} \leq \rho$, simply note that if $\varphi_{\infty} \not \leq \rho$ then there would exist $i \in \N$ and some $x \in E_i$ such that:
                    $$\varphi_{\infty}|_{E_i}(x) := \varphi_i(x) > \rho(x)$$
                which is contradictory to the fact that every $\varphi_i$ is dominated by $\rho$.
            \end{proof}
        \begin{corollary}[Complex Hahn-Banach and norm extension] \label{coro: norm_extensions}
            Now, let $K$ be an arbitrary complete archimedean field again, and let $E$ be a $K$-vector space and let $\varphi_0: E_0 \to K$ be a $K$-linear functional defined on some subspace $E_0 \subseteq E$. Suppose also that $\rho: E \to \R_{\geq 0}$ is a semi-norm. If:
                $$\forall x \in E_0: |\varphi_0(x)| \leq \rho(x)$$
            then there shall exist a linear extension:
                $$\varphi_{\infty}: E \to \R$$
            of $\varphi_0$ which is such that:
                $$\forall x \in E: |\varphi_{\infty}(x)| \leq \rho(x)$$

            In particular, if $E$ is normed and $\rho$ is given by $\rho(x) := \norm{\varphi_0}_{E^*_{\cont}} \norm{x}_E$ then:
                $$\norm{\varphi_0}_{E^*_{\cont}} = \norm{\varphi_{\infty}}_{E^*_{\cont}}$$
        \end{corollary}
            \begin{proof}
                If $K = \R$ then the assertion comes directly from the fact that $\varphi_0(x) \leq |\varphi_0(x)|$ and from theorem \ref{theorem: hahn_banach}.

                If $K = \bbC$ then we can reduce the assertion to the real case by writing $\varphi_0 := \Re(\varphi_0) + i \Im(\varphi_0)$; by linearity, we even have that:
                    $$\Im(\varphi_0) = \Re(i \varphi_0)$$
            \end{proof}

        A somewhat non-trivial consequence of the Hahn-Banach Theorem is that the category $K\-\Ban$ is enriched over itself (see lemma \ref{lemma: hom_between_banach_spaces_are_banach_spaces}), i.e. the hom-space between two Banach spaces is once again a Banach space. This also requires the Uniform Boundedness Principle (theorem \ref{theorem: uniform_boundedness}) as a prerequiresite. 
        \begin{lemma}[Domain completion\footnote{If only I could call this the \say{Domain Expansion Lemma} ...}] \label{lemma: domain_completion}
            If $X, Y$ are normed spaces then $\Hom_{K, \cont}(X, Y)$. Denote by $X^{\wedge}$ the completion of $X$. Then, there shall be a short isomorphism:
                $$\Phi_Y: \Hom_{K, \cont}(X^{\wedge}, Y) \xrightarrow[]{\cong} \Hom_{K, \cont}(X, Y)$$
            that is natural in $Y$.
        \end{lemma}
            \begin{proof}
                \todo[inline]{Not done}
            \end{proof}
        \begin{lemma}[The category of Banach spaces is self-enriching] \label{lemma: hom_between_banach_spaces_are_banach_spaces}
            If $X, Y$ are normed spaces then $\Hom_{K, \cont}(X, Y)$ will be a Banach space if and only if $Y$ is a Banach space. Consequently, the category $K\-\Ban$ is self-enriching.
        \end{lemma}
            \begin{proof}
                By lemma \ref{lemma: domain_completion}, we can assume without loss of generality that $X$ is Banach.
                
                Assume firstly that $Y$ is Banach. Then, any Cauchy sequence $\{T_n\}_{n \geq 0} \subset \Hom_{K, \cont}(X, Y)$ will be pointwise convergent in $Y$. By corollary \ref{coro: compact_limits_of_pointwise_convergent_sequences_of_continuous_linear_maps}, there exists a uniform limit $\{T_n\}_{n \geq 0} \to T$, thus showing that any Cauchy sequence in $\Hom_{K, \cont}(X, Y)$ converges with respect to $\norm{-}_{\Hom_{K, \cont}(X, Y)}$ The normed space $\Hom_{K, \cont}(X, Y)$ is therefore Banach.

                Conversely, suppose that $\Hom_{K, \cont}(X, Y)$ is Banach. For each $x \in X$, consider the evaluation map:
                    $$\ev_x: \Hom_{K, \cont}(X, Y) \to Y$$
                given by:
                    $$\ev_x[T] := T(x)$$
                which clearly makes it linear. We claim that $\ev_x$ is also continuous and surjective, thus making $Y$ a Banach space with respect to the quotient topology inherited from $\Hom_{K, \cont}(X, Y)$.
                \begin{itemize}
                    \item Since $\Hom_{K, \cont}(X, Y)$ is Banach, any Cauchy sequence $\{T_n\}_{n \geq 0} \subset \Hom_{K, \cont}(X, Y)$ will be convergent with respect to $\norm{-}_{\Hom_{K, \cont}(X, Y)}$, say to some $T \in \Hom_{K, \cont}(X, Y)$ (which is unique because $\Hom_{K, \cont}(X, Y)$ is Hausdorff by virtue of being a metric space). Consequently, $\{T_n(x)\}_{n \geq 0} \to T(x)$ with respect to $\norm{-}_Y$, and hence:
                        $$\lim_{n \to +\infty} \ev_x[T_n] = \lim_{n \to +\infty} T_n(x) = T(x) = \ev_x[T] = \ev_x\left[ \lim_{n \to +\infty} T_n \right]$$
                    from which we gather that $\ev_x$ is continuous.
                    \item Next, to prove that $\ev_x$ is surjective, pick an arbitrary $\phi \in X^*_{\cont}$, and then for each $y \in Y$, note that there exists $\Phi_y \in \Hom_{K, \cont}(X, Y)$ given by:
                        $$\Phi_y(x) := \phi(x) y$$
                    for every $x \in X$.
                \end{itemize}
            \end{proof}
        \begin{corollary}[Duals of normed spaces are Banach] \label{coro: duals_of_normed_space_are_banach}
            Let $X$ be normed space and let $X^*_{\cont} := \Hom_{K, \cont}(X, K)$. Then $X^*_{\cont}$ will be a Banach space, regardless of whether or not $X$ is Banach.
        \end{corollary}
            \begin{proof}
                Clear from lemma \ref{lemma: hom_between_banach_spaces_are_banach_spaces} and the fact that $K$ is a Banach space over itself by assumption.
            \end{proof}
        \begin{remark}
            By the Hahn-Banach Theorem (theorem \ref{theorem: hahn_banach}), $X^*_{\cont}$ is non-zero if and only if $X$ is non-zero, so corollary \ref{coro: duals_of_normed_space_are_banach} is a rather non-trivial result.
        \end{remark}

        Another (easy) corollary of the Hahn-Banach Theorem is that there is a non-trivial (and representable) duality functor:
            $$(-)^* := \Hom_K(-, \R): K\-\Vect \to K\-\Vect^{\op}$$
        with a non-trivial restriction:
            $$(-)^*_{\cont} \cong \Hom_{K, \cont}(-, \R): K\-\Vect_{\norm{-}} \to K\-\Vect_{\norm{-}}^{\op}$$
        down to the subcategory of normed vector spaces and \textit{continuous} linear maps between them, called the \textbf{continuous duality} functor. This restriction maps normed vector spaces $E$ to $E^*_{\cont}$ equipped with the sup-norm. Again, let us remark that even though the functors $(-)^*$ and $(-)^*_{\cont}$ do exist abstractly (as they are just contravariant hom-functors), we do not know before the Hahn-Banach Theorem as to whether or not they might have reasonably large essential images.
        
        The following definition is tautologically equivalent to the definition of contravariant hom-functors, though we state it regardless for the sake of establishing the terminologies. 
        \begin{definition}[Transposition] \label{def: transposition}
            Given a (continuous) $K$-linear map $T: E \to E'$, we call its image $T^*: E'^* \to E^*$ (respectively, $T^*_{\cont}: E'^*_{\cont} \to E^*_{\cont}$) under the (continuous) duality functor the \textbf{(continuous) transpose} of $T$. Explicitly, this is given by:
                $$T^*[\psi](x) := \psi( T(x) )$$
            for all $\psi \in E'^*$ and all $x \in E$ (and likewise for $T^*_{\cont}$).
        \end{definition}

        Let us now investigate the properties of the functor $(-)^*_{\cont}$.
        \begin{convention}
            To avoid notation clutter, the self-composition of $(-)^*_{\cont}$ shall be denoted by $(-)^{**}_{\cont}$.
        \end{convention}
        \begin{lemma}[Continuous duality is involutive] \label{lemma: continuous_duality_involutive}
            The continuous duality functor $(-)^*_{\cont}$ is involutive, i.e. $(-)^{**}_{\cont}$ is the identity functor. 
        \end{lemma}
            \begin{proof}
                Firstly, we shall need to prove that there is a short isomorphism $E \xrightarrow[]{\cong} ( E^*_{\cont} )^*_{\cont}$ for any normed vector space $E$. We claim that this is given by:
                    $$x \mapsto (E^*_{\cont} \xrightarrow[]{\ev_x} \R)$$
                where:
                    $$\ev_x[\varphi] := \varphi(x)$$
                To see that this is a short isomorphism, simple note that the following holds for all $x \in E$ and all $\varphi \in E^*_{\cont}$:
                    $$\norm{\ev_x}_{\sup} := \sup_{\varphi \in E^*_{\cont}, \norm{\varphi}_{E^*_{\cont}} = 1} |\ev_x[\varphi]| = \sup_{\varphi \in E^*_{\cont}, \norm{\varphi}_{E^*_{\cont}} = 1} \abs{\varphi(x)} = \norm{x}_E$$

                Next, let $E, E'$ be normed spaces and denote the canonical short isomorphisms between them and their continuous double duals by $\ev$ and $\ev'$ respectively. Let $T: E \to E'$ be a continuous linear map. Then, precisely because $\ev$ and $\ev'$ are isometric, we have the following:
                    $$\norm{ T^{**}_{\cont} }_{\sup} = \norm{ \ev' \circ T \circ \ev^{-1} }_{\sup} := \sup_{y \in E^{**}_{\cont}} \frac{ \norm{ (\ev' \circ T \circ \ev^{-1})(y) }_{\sup} }{\norm{y}_E} = \sup_{y \in E^{**}_{\cont}} \frac{\norm{T(y)}_{E'}}{\norm{y}_E} = \norm{T}_{\sup}$$
                This concludes the proof that there is a isometry:
                    $$\Hom_{K, \cont}(E, E') \xrightarrow[]{\cong} \Hom_{K, \cont}(E^{**}_{\cont}, E'^{**}_{\cont})$$
            \end{proof}
        For the proof of the following proposition, it will be useful to note that if $\Phi: V \to V'$ is any linear map between normed spaces, then:
            $$\forall v \in V: \norm{\Phi}_{V^*} := \sup_{v \in V} \frac{\norm{\Phi(v)}}{\norm{v}_V} \implies \norm{\Phi}_{V^*_{\cont}} \norm{v}_V \geq \abs{\Phi(v)}$$
        \begin{proposition}[Norms of transpositions] \label{prop: transposition_norms}
            Let $T: E \to E'$ be any continuous linear map between normed spaces. Then:
                $$\norm{T}_{\sup} = \norm{T^*_{\cont}}_{\sup}$$
            In fact, the continuous duality functor $(-)_{\cont}^*$ is not only fully faithful, but also short-isomorphic on hom-spaces, in the sense that each of the $K$-linear isomorphisms:
                $$\Hom_{K, \cont}(E, E') \xrightarrow[]{\cong} \Hom_{K, \cont}(E'^*_{\cont}, E^*_{\cont})$$
            is moreover a short isomorphism, for all $E, E' \in \Ob( K\-\Vect_{\norm{-}} )$.
        \end{proposition}
            \begin{proof}
                For any $\psi \in E'^*_{\cont}$ and any $x \in E$, we have that:
                    $$\abs{ T^*_{\cont}[\psi](x) } = \abs{\psi(T(x))} \leq \norm{\psi}_{E^*_{\cont}} \norm{T(x)}_{E'} \leq \norm{\psi}_{E'^*_{\cont}} \norm{T}_{\sup} \norm{x}_E$$
                and hence:
                    $$\forall x \in E: \forall \psi \in E'^*_{\cont}: \norm{T}_{\sup} \geq \frac{ \abs{ T^*_{\cont}[\psi](x) } }{\norm{\psi}_{E'^*_{\cont}} \norm{x}_E }$$
                from which one gathers that $\norm{T}_{\sup} \geq \norm{T^*_{\cont}}_{\sup}$. Arguing similarly will yield us $\norm{T^*_{\cont}}_{\sup} \geq \norm{T^{**}_{\cont}}_{\sup}$. But we know from lemma \ref{lemma: continuous_duality_involutive} that $\norm{T^{**}_{\cont}}_{\sup} = \norm{T}_{\sup}$, and hence $\norm{T}_{\sup} \leq \norm{T^*_{\cont}}_{\sup}$ as well, and as such we have shown that $\norm{T}_{\sup} = \norm{T^*_{\cont}}_{\sup}$.
            \end{proof}
        \begin{proposition}[Properties of continuous duals] \label{prop: properties_of_continuous_duals}
            The continuous duality functor $(-)^*_{\cont}$ preserves the following (co)limits in $K\-\Vect_{\norm{-}}$:
            \begin{enumerate}
                \item finite direct sums;
                \item short exact sequences:
                    $$0 \to F \xrightarrow[]{\ker \pi} E \xrightarrow[]{\pi} Q \to 0$$
                where $F \subseteq E$ is a closed subspace, and in fact, we have a short isomorphism:
                    $$T: \Ann_{E^*_{\cont}}(F) \xrightarrow[]{\cong} Q^*_{\cont}$$
                determined by:
                    $$\forall x \in E: T[\varphi]( \pi(x) ) := \varphi(\pi(x))$$
                where $\Ann_{E^*}(F) := \{ \varphi \in E^* \mid \varphi(F) = 0 \}$, equipped with the subspace topology; note that there is a small abuse of notations here: the domain of $\varphi$ is not spanned by the vectors $\pi(x) \in Q$, but rather their images under some linear splitting\footnote{Which always exists.} $Q \to E$ of the quotient map $\pi: E \to Q$.
            \end{enumerate}
        \end{proposition}
            \begin{proof}
                \begin{enumerate}
                    \item This is self-evident.
                    \item The definition of the quotient topology ensures that the quotient map $\pi: E \to Q$ is continuous, and so in particular, it preserves convergence of sequences. $\pi$ is also surjective, so any sequence in $Q$ arises as the image of a sequence in $E$. Closedness of $Q^*_{\cont}$ inside $E_{\cont}^*$ via $\pi_{\cont}^*$ then follows from the fact that $\norm{\pi^*_{\cont}}_{\sup} = \norm{\pi}_{\sup}$, as proven in proposition \ref{prop: transposition_norms}, and hence the functor $(-)^*_{\cont}$ maps a short exact sequence:
                        $$0 \to F \xrightarrow[]{\ker \pi} E \xrightarrow[]{\pi} Q \to 0$$
                    in $K\-\Vect_{\norm{-}}$ wherein $F \subseteq E$ is a closed subspace, to the following short exact sequence in $K\-\Vect_{\norm{-}}^{\op}$:
                        $$0 \to Q^*_{\cont} \xrightarrow[]{\pi^*_{\cont}} E^*_{\cont} \xrightarrow[]{\coker \pi^*_{\cont}} F^*_{\cont} \to 0$$
                    wherein $Q^*_{\cont} \subseteq E^*_{\cont}$ is a closed subspace.

                    Lastly, to prove that $T: \Ann_{E^*_{\cont}}(F) \to Q^*_{\cont}$ is a short isomorphism, consider the following for any $\varphi \in \Ann_{E^*_{\cont}}(F)$:
                        $$\norm{T[\varphi]}_{Q^*_{\cont}} := \sup_{x \in E} \frac{ \abs{T[\varphi]( \pi(x) )} }{ \norm{\pi(x)}_Q } = \sup_{x \in E} \frac{ \abs{\varphi(\pi(x))} }{ \norm{\pi(x)}_Q } = \norm{\varphi}_{\Ann_{E^*_{\cont}}(F)}$$
                    wherein the last equality holds because $\varphi(F) = 0$ for all $\varphi \in \Ann_{E^*_{\cont}}(F)$. Linearity of $T$ is self-evident.
                \end{enumerate}
            \end{proof}
        \begin{corollary}
            Let:
                $$0 \to F \xrightarrow[]{\ker \pi} E \xrightarrow[]{\pi} Q \to 0$$
            be a short exact sequence in $K\-\Vect_{\norm{-}}$, wherein $F \subseteq E$ is a closed subspace. Then there is a short isomorphism:
                $$F^*_{\cont} \xrightarrow[]{\cong} E^*_{\cont}/\Ann_{E^*_{\cont}}(F)$$
                $$\psi \mapsto \psi|_F$$
        \end{corollary}
        \begin{proposition}[Split monomorphisms of normed spaces: linear closed immersions retract] \label{prop: split_monomorphisms_of_normed_spaces}
            The category $K\-\Vect_{\norm{-}}$ in fact also admits split monomorphisms, which are precisely linear closed immersions.
        \end{proposition}
            \begin{proof}
                If $E$ is any normed space and $F$ is any subspace therein, then:
                    $$E^*_{\cont} \cong F^*_{\cont} \oplus \Ann_{E^*_{\cont}}(F)$$
                by the definition of $\Ann_{E^*_{\cont}}(F)$ as in proposition \ref{prop: properties_of_continuous_duals}. Next, let us consider a short exact sequence in $K\-\Vect_{\norm{-}}$:
                    $$0 \to F \xrightarrow[]{\ker \pi} E \xrightarrow[]{\pi} Q \to 0$$
                wherein $F \subseteq E$ is a closed subspace. In this situation, we now know that there is a short isomorphism:
                    $$T: \Ann_{E^*_{\cont}}(F) \xrightarrow[]{\cong} Q^*_{\cont}$$
                and hence there is an isomorphism of normed spaces:
                    $$\id_{F^*_{\cont}} \oplus T: F^*_{\cont} \oplus \Ann_{E^*_{\cont}}(F) \xrightarrow[]{\cong} F^*_{\cont} \oplus Q^*_{\cont}$$
                From this, we see that there is a section:
                    $$s: F^*_{\cont} \to E^*_{\cont}$$
                of $\coker \pi^*_{\cont}: E^*_{\cont} \to F^*_{\cont}$ given by inclusion into the first direct summand. Now, since $(-)^*_{\cont}$ is involutive (see lemma \ref{lemma: continuous_duality_involutive}), we can apply the functor once more to get a linear continuous retract:
                    $$s^*_{\cont}: E \to F$$
                (note the implicit identifications of $E, F$ with the continuous double duals).
            \end{proof}
        \begin{corollary}[Annihilators vs. orthogonal complements] \label{coro: annihilators_vs_orthogonal_complements}
            Let $E$ be a normed space and $F \subseteq E$ be a closed vector subspace thereof. Then, there shall be a short isomorphism:
                $$F^{\perp}_{\cont} \cong ( \Ann_{E^*_{\cont}}(F) )^*_{\cont}$$
        \end{corollary}
            \begin{proof}
                The definition of annihlators guarantee us that:
                    $$E^*_{\cont} \cong F^*_{\cont} \oplus \Ann_{E^*_{\cont}}(F)$$
                Since we also know that $(-)^*_{\cont}$ is involutive (see lemma \ref{lemma: continuous_duality_involutive}), meaning that there are short isomorphisms between $E, F$ and their continuous double duals $E^{**}_{\cont}, F^{**}_{\cont}$, continuously dualising the isomorphism above shall yield\footnote{We are also using the fact that $(-)^*_{\cont}$ preserves finite direct sums (see proposition \ref{prop: properties_of_continuous_duals}).}:
                    $$E \cong F \oplus \Ann_{E^*_{\cont}}(F)^*_{\cont}$$
                and hence there is a short isomorphism:
                    $$F^{\perp}_{\cont} \cong ( \Ann_{E^*_{\cont}}(F) )^*_{\cont}$$
                wherein the LHS is the orthogonal complement $F^{\perp}$ of $F$ inside $E$, equipped with the subspace topology inherited from the norm topology on $E$.
            \end{proof}
        \begin{remark}[Orthogonal complements are continuous sections]
            Let $E$ be a normed space and $F \subseteq E$ be a closed vector subspace thereof.
        
            Note that by construction, $\Ann_{E^*_{\cont}}(F)$ is a closed vector subspace of $E^*$ (for any vector subspace $F \subseteq E$ actually, not just closed ones) and fits into the following short exact sequence:
                $$0 \to \Ann_{E^*_{\cont}}(F) \to E^*_{\cont} \to F^*_{\cont} \to 0$$
            where the arrow $E^*_{\cont} \to F^*_{\cont}$ is the image under $(-)^*_{\cont}$ of the inclusion $F \hookrightarrow E$. This, in turn, gives rise to a short exact sequence:
                $$0 \to F \to E \to ( \Ann_{E^*_{\cont}}(F) )^*_{\cont} \to 0$$
            Since $\Ann_{E^*_{\cont}}(F)$ is a closed subspace of $E^*_{\cont}$, its inclusion into $E^*_{\cont}$ admits a retract $E^*_{\cont} \to \Ann_{E^*_{\cont}}(F)$, which gives rise to a section $( \Ann_{E^*_{\cont}}(F) )^*_{\cont} \to E$. The image of this section is nothing but $F^{\perp}_{\cont}$, per the definition of orthogonal complements. As such, we see that corollary \ref{coro: annihilators_vs_orthogonal_complements} really is a consequence of the fact that $K\-\Vect_{\norm{-}}$ admits split monomorphisms, as shown in proposition \ref{prop: split_monomorphisms_of_normed_spaces}.
        \end{remark}
        
        \begin{lemma}[A density criterion] \label{lemma: density_orthogonal_complement_criterion}
            Let $E$ be a normed space. A vector subspace $W \subseteq E$ is dense if and only if $W^{\perp}_{\cont} \cong 0$.
        \end{lemma}
            \begin{proof}
                Suppose firstly that we have a dense vector subspace $W \subseteq E$. This is to say that any point $x \in E$ is the limit of some Cauchy sequence $\{w_n\}_{n \geq 0} \subset W$. Then, for all $\varphi \in \Ann_{E^*}(W)$, we shall have the following as a consequence of $\varphi$ being continuous:
                    $$\varphi(x) = \lim_{n \to +\infty} \varphi(w_n) = 0$$
                But this implies that $\varphi(x) = 0$ for all $x \in E$, i.e. $\varphi = 0$. From this, we infer that:
                    $$\Ann_{E^*}(W) \cong 0$$
                which then implies that:
                    $$W^{\perp}_{\cont} \cong 0$$
                by duality (cf. corollary \ref{coro: annihilators_vs_orthogonal_complements}).

                Conversely, if $W^{\perp}_{\cont} \cong 0$, then $W = E$, and hence $\overline{W} = E$ trivially, which means that $W$ is tautologically dense inside $E$.
            \end{proof}

        Now, let $E$ be a normed space and $E^*$ be its algebraic linear dual (without any topology equipped for now). There is an evident bilinear pairing:
            $$\<-, -\>: E^* \tensor_K E \to K$$
        given by:
            $$\forall (\varphi, x) \in E^* \x E: \<\varphi, x\> := \varphi(x)$$
        \begin{question}
            What is the coarsest possible topology (i.e. as few open subsets as possible) that one can equip $E^*$ with so that its elements are continuous as functions $E \to K$ between normed spaces ?
        \end{question}
        A linear functional $\varphi \in E^*$ is continuous with respect to the norm topologies on $E$ and $K$ if and only if for any sequence $\{x_n\}_{n \geq 0} \subset E$ with limit $x \in E$:
            $$\forall \e > 0: n \gg 0 \implies \abs{ \varphi(x_n) - \varphi(x) } < \e$$
        As $\varphi: E \to K$ is linear, we can write:
            $$\abs{ \varphi(x_n) - \varphi(x) } = \abs{\varphi(x_n - x)}$$
        and hence observe that:
            $$\abs{\varphi(x_n - x)} \leq \norm{\varphi}_{E^*_{\cont}} \norm{x_n - x}_E$$
        for every convergent sequence $\{x_n\}_{n \geq 0} \to x$, since $\norm{\varphi}_{E^*_{\cont}} := \sup_{y \in E} \frac{\abs{\varphi(y)}}{\norm{y}_E}$. Due to the convergence $\{x_n\}_{n \geq 0} \to x$, we also have that:
            $$\forall \e > 0: n \gg 0 \implies \norm{x_n - x}_E < \e$$
        and so to ensure that $\abs{\varphi(x_n) - \varphi(x)} < \e$ for all $\e$ and all $n \gg 0$, we shall need to require that:
            $$\norm{x_n - x}_E < \frac{\e}{\norm{\varphi}_{E^*_{\cont}}}$$
        (the functional $0$ is tautologically continuous, and for all other cases, the RHS is well-defined); note also that $\varphi$ is continuous if and only if it is bounded, so the RHS never vanishes. Since $\e$ is constant, the LHS increases as $\norm{\varphi}_{E^*_{\cont}}$ decreases, and hence $\{\varphi(x_n)\}_{n \geq 0} \to \varphi(x)$ as soon as $x_n$ lies within a relatively \say{large} open ball centered at $x$. The sought-for topology on $E^*$ is thus indeed coarser than the topology induced by the sup-norm $\norm{-}_{E^*_{\cont}}$; in comparison to the former, we may refer to the latter as the \textbf{strong topology}.
        \begin{definition}[Weak topologies] \label{def: weak_topologies}
            Let $E$ be a normed space. A \textbf{weak topology} on $E^*$ is a topology such that any $\varphi \in E^*$, when regarded as a function $\varphi: E \to K$, is continuous with respect to the norm topologies on $E$ and $K$.
        \end{definition}
        \begin{proposition}[Universal property of the weak topology]
            Let $E$ be a normed space. Any weak topology on $E^*$ is actually initial amongst all topologies on $E^*$ such that any $\varphi \in E^*$ is continuous, i.e. any other topology - which is a certain subset of $\calP(E^*)$ - satisfying the condition above contains any weak topology as a subset. Consequently, the weak topology is uniquely defined.
        \end{proposition}
            \begin{proof}
                Let $\calW$ be the set whose elements $w \in \calW$ enumerate all topologies $\tau_w$ on $E^*$ such that any $\varphi \in E^*$ is continuous. Next, pick an arbitrary functional $\varphi \in E^*$ and an open subset $V \subseteq K$. The functional $\varphi$ is continuous with respect to some topology $\tau_w$ if and only if:
                    $$\exists w \in \calW: \varphi^{-1}(V) \in \tau_w$$
                which implies that:
                    $$\forall \varphi \in E^*: \varphi^{-1}(V) \in \weak \iff \varphi^{-1}(V) \in \bigcap_{w \in \calW} \tau_w \iff ( \forall w \in \calW: \varphi^{-1}(V) \in \tau_w )$$
                The rest of the claim then follows.
            \end{proof}
        \begin{convention}
            Let $E$ be a normed space. When $E^*$ is equipped with the weak topology, we shall denote it by $E^*_{\weak}$. 
            
            If a sequence of points $\{x_n\}_{n \geq 0}$, which is not necessarily convergent, is such that $\{\varphi(x_n)\}_{n \geq 0} \to \varphi(x)$ for some $x \in E$ and for all functionals $\varphi \in E^*$, then we will say that the sequence $\{x_n\}_{n \geq 0}$ \textbf{converges weakly} to $x$ and write:
                $$\{x_n\}_{\geq 0} \xrightarrow[]{\weak} x$$
        \end{convention}
        \begin{example}[$L^p$-convergence]
            For more details on $L^p$-spaces, see subsection \ref{subsection: L_p_spaces}.

            Let $(X, \mu)$ be a measure space and let $p, q \in \N_{\geq 1} \cup \{+\infty\}$ be such that $\frac1p + \frac1q = 1$. Then, by theorem \ref{theorem: L_p_space_duality}, we have that:
                $$L^p(X, \mu)^*_{\weak} \cong L^q(X, \mu)$$
            or in formulae, we have that:
                $$\{f_n\}_{n \geq 0} \xrightarrow[]{\weak} f \iff \left( \forall g \in L^q(X, \mu): \left\{ \int_X f_n g d\mu \right\}_{n \geq 0} \to \int_X fg d\mu \right)$$
            for $f_n, f \in L^p(X, \mu)$.
        \end{example}
        \begin{convention}[The weak-$*$ topology]
            There is also something called the \say{weak-$*$ topology}, which is nothing more than the construction of the weak topology on $E^{**}$, regarded as the (algebraic) dual of the normed space $E^*_{\cont}$. We will refrain from using this terminology, as it is somewhat awkward.
        \end{convention}
        Despite being rather coarse, the weak topology is still sufficiently fine and not too pathological.
        \begin{lemma}[Weak topology is Hausdorff] \label{lemma: weak_topology_is_hausdorff}
            For any normed space $E$, the weak topology on $E^*$ is Hausdorff.
        \end{lemma}
            \begin{proof}
                \todo[inline]{Intersection/coarsenings of Hausdorff topologies are not necessarily Hausdorff, since there may not be enough open sets to separate points.}
            \end{proof}
        \begin{theorem}[Weak duality for finite-dimensional normed spaces] \label{theorem: finite_dimensional_weak_duality}
            Let $E$ be a normed space. Then, the weak topology on $E^*$ will coincide with the sup-norm topology (i.e. weak = strong) if and only if $E$ is finite-dimensional.
        \end{theorem}
            \begin{proof}
                
            \end{proof}
        \begin{theorem}[Banach-Alaoglu: bounded sequences converge weakly] \label{theorem: banach_alaoglu}
            Let $E$ be a normed space. Then $E^{**}_{\weak}$ will be compact.
        \end{theorem}
            \begin{proof}
                
            \end{proof}